\documentclass[a4paper,
fontsize=11pt,
%headings=small,
oneside,
numbers=noperiodatend,
parskip=half-,
bibliography=totoc,
final
]{scrartcl}

\usepackage{synttree}
\usepackage{graphicx}
\setkeys{Gin}{width=.6\textwidth} %default pics size

\graphicspath{{./plots/}}
\usepackage[ngerman]{babel}
%\usepackage{amsmath}
\usepackage[utf8x]{inputenc}
\usepackage [hyphens]{url}

\usepackage[colorlinks, linkcolor=black,citecolor=black, urlcolor=blue,
breaklinks= true]{hyperref}
\usepackage{booktabs} 
\usepackage[left=2.4cm,right=2.4cm,top=2.3cm,bottom=2cm,includeheadfoot]{geometry}
\usepackage{eurosym}
\usepackage{multirow}
\usepackage[ngerman]{varioref}
\setcapindent{1em}
\renewcommand{\labelitemi}{--}
\usepackage{paralist}
\usepackage{pdfpages}
\usepackage{lscape}
\usepackage{float}
\usepackage{acronym}
\usepackage{eurosym}
\usepackage[babel]{csquotes}
\usepackage{longtable,lscape}
\usepackage{mathpazo}
\usepackage[flushmargin,ragged]{footmisc} % left align footnote

\urlstyle{same}  % don't use monospace font for urls

\usepackage[fleqn]{amsmath}

%adjust fontsize for part

\usepackage{sectsty}
\partfont{\large}

%Das BibTeX-Zeichen mit \BibTeX setzen:
\def\symbol#1{\char #1\relax}
\def\bsl{{\tt\symbol{'134}}}
\def\BibTeX{{\rm B\kern-.05em{\sc i\kern-.025em b}\kern-.08em
    T\kern-.1667em\lower.7ex\hbox{E}\kern-.125emX}}

\usepackage{fancyhdr}
\fancyhf{}

\pagestyle{fancyplain}

\fancyhead[R]{\thepage}
%meta

\fancyhead[L]{J. Schöpfel \\ % author
LIBREAS. Library Ideas, 24 (2014) % journal, issue, volume
\href{http://nbn-resolving.de/urn:nbn:de:kobv:11-100215909}{urn:nbn:de:kobv:11-100215909}} %urn
\fancyfoot[L] {\textit{Creative Commons BY 3.0}} %licence
\fancyfoot[R] {\textit{ISSN: 1860-7950}}

\title{\LARGE{Note de lecture} \\ \large{Rezension zu: L’avenir des bibliothèques: l’exemple des bibliothèques
universitaires \/ Florence Roche et Frédéric Saby (Hg.). – Villeurbanne:
Presses de l’enssib, 2013. – 224 Seiten. – ISBN 978-10-91281-13-3: \EUR{34}}} %title
\author{Joachim Schöpfel} %author

\date{}

\begin{document}

\maketitle
\thispagestyle{fancyplain} 





Woody Allen sagte einmal: \enquote{Ich denke viel an die Zukunft, weil
das der Ort ist, wo ich den Rest meines Lebens zubringen werde.} Die
französische Fachhochschule für Bibliothekswesen ENSSIB in
Lyon\footnote{Ecole Nationale Supérieure des Sciences de l'Information
  et des Bibliothèques~ \url{http://www.enssib.fr/}} hat in diesem Sinn
im letzten Jahr ein Buch herausgegeben, das die Zukunft der
Universitätsbibliotheken zum Thema hat. Es handelt sich um eine
theoretische Studie, die die Herausforderungen, Ursachen und Geschichte
des Themas herausarbeitet, es aber dem Leser überlässt, daraus die
praktischen Schlussfolgerungen zu ziehen.

Die zentrale Frage lautet: \enquote{Wie kann die Bibliothek zur
wissenschaftlichen Exzellenz der Universität beitragen?} (S. 9,
Übersetzung des Autors) Welchen Mehrwert steuert sie für Lehre und
Forschung auf dem Campus bei? Von Anfang an beschränkt sich das Buch vor
allem auf einen Bereich, auf die Funktion und die Qualität der
Dienstleistungen für ein Publikum -- Studenten, Lehrkräfte und
Wissenschaftler --, welches sich immer mehr von der Bibliothek als
Einrichtung entfernt. Das Kredo der Autoren, welches das Buch wie ein
roter Faden durchzieht, lautet demzufolge auch: \enquote{Es geht darum,
eine neue Beziehung herzustellen} (S. 10, Übersetzung des Autors).

Die ersten sechs Kapitel beschreiben das Umfeld. Was tut sich auf dem
Campus, wie verändert sich die Einrichtung Universität? Die Reformen des
französischen Hochschulwesens haben die Verwaltung, die Strukturen und
die Kompetenzen der Universitäten in Bewegung kommen lassen. Diese
Dynamik neuer Aufgaben, neuer Inhalte und Zielgruppen betrifft auch die
Bibliothek. Wie hat sich das Verhältnis der Studenten zu Bildung und
Wissen verändert? Wie gehen sie mit dem Angebot der Bibliotheken um? Wie
sehen ihre Bedürfnisse in Sachen Information und Dokumentation aus?

Diese ersten Kapitel untersuchen ebenfalls, wie man eine Politik
öffentlicher Dienstleistungen für diese neuen Zielgruppen definiert, wie
man sie evaluiert und wie man die institutionelle Strategie in
Kommunikation umsetzt. Ein anderes Kapitel betrifft die Entwicklung der
Raumnutzung und der Innenarchitektur, insbesondere im Hinblick auf die
neuen Learning Centers.

Die drei folgenden Kapitel beschreiben Zukunftsszenarien der
Universitätsbibliotheken unter verschiedenen Blickwinkeln. Dabei geht es
um die Zukunft der Bibliothek als Ort, um die Zukunftsaussichten der
Bibliothekare als Berufsgruppe und um die Zukunft der Bibliotheks- und
Informationswissenschaft. Die Autoren legen hier den Schwerpunkt auf
drei unterschiedliche Aspekte: auf das neue Konzept der Learning
Centers, \enquote{Orte der Konvergenz und Verankerung}; auf die
Kapazität der Bibliothekare, ihren Beruf aus dem Verständnis und der
Beziehung zum Publikum heraus zu konstruieren; und auf die Entwicklung
der Bibliothek als Ort öffentlicher Dienstleistungen am Publikum.

Das Buch fasst die unterschiedlichen Studien wie folgt zusammen:
\enquote{Die örtlichen Bibliotheken werden weiterhin existieren, aber
nur unter der Bedingung, dass sie ihren Ansatz radikal verändern und
sich auf die Dienstleistungen am Publikum konzentrieren, auf Kosten der
Bestandserhaltung und -entwicklung} (S. 207, Übersetzung des Autors).
Den Autoren des Buches -- Wissenschaftler und Bibliothekare aus Grenoble
-- kommt das Verdienst zu, die Zukunft der Universitätsbibliotheken
unter dem Gesichtspunkt ihrer Benutzer (und Nicht-Benutzer) zu
beschreiben, mit Dienstleistungen, die den Bedürfnissen der Studenten,
Lehrkräfte und Wissenschaftler entsprechen. Manchem Leser mag eine
weitergehende politische oder wirtschaftliche Analyse fehlen. Die
Universitätsbibliothek hat andere Rollen, sowohl im Bereich von
Wissenschaftsbewertung und Informationsmanagement (Szientometrie) als
auch in der Verbreitung wissenschaftlicher Veröffentlichungen (Open
Access und so weiter), bei der Öffentlichkeitsarbeit und in Kontakten zu
Vereinen, lokalen Gruppen et ceterea.~

Das Buch richtet sich in erster Linie an Universitätsbibliothekare sowie
an Studenten und Forscher der Bibliotheks- und Informationswissenschaft.
\enquote{Prognosen sind schwierig, besonders wenn sie die Zukunft
betreffen}, soll der dänische Physiker Niels Bohr einmal in einer Rede
gesagt haben. Besser in der Analyse als in der Prognose liefert dieses
neue Buch aus dem ENSSIB-Verlag dennoch einen interessanten Beitrag zur
Debatte über die Zukunft der Universitätsbibliotheken.

Für den deutschsprachigen Leser bietet dieses Buch die doppelte
Gelegenheit, ein besseres Verständnis für die tiefgreifenden
Veränderungen im französischen Hochschulwesen zu gewinnen und zu
beobachten, wie sich neue und globale Konzepte, beispielsweise das
Learning Center, an örtliche Gegebenheiten anpassen können und müssen.
Zugleich kann er/sie auch seine/ihre französischen Sprachkenntnisse im
Bereich der Bibliotheks- und Informationswissenschaft vertiefen, dies
auch und vor allem im Hinblick auf die diesjährige IFLA-Konferenz in
Lyon\ldots{}

\begin{center}\rule{3in}{0.4pt}\end{center}

\textbf{Joachim Schöpfel} ist leitender Direktor des Atelier National de
Reproduction des Thèses (ARNT) und Dozent am Fachbereich
Informationswissenschaften der Université Charles-de-Gaulle Lille 3.
Kontakt: joachim.schopfel@univ-lille3.fr

\end{document}