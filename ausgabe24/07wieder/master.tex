\documentclass[a4paper,
fontsize=11pt,
%headings=small,
oneside,
numbers=noperiodatend,
parskip=half-,
bibliography=totoc,
final
]{scrartcl}

\usepackage{synttree}
\usepackage{graphicx}
\setkeys{Gin}{width=.6\textwidth} %default pics size

\graphicspath{{./plots/}}
\usepackage[ngerman]{babel}
%\usepackage{amsmath}
\usepackage[utf8x]{inputenc}
\usepackage [hyphens]{url}

\usepackage[colorlinks, linkcolor=black,citecolor=black, urlcolor=blue,
breaklinks= true]{hyperref}
\usepackage{booktabs} 
\usepackage[left=2.4cm,right=2.4cm,top=2.3cm,bottom=2cm,includeheadfoot]{geometry}
\usepackage{eurosym}
\usepackage{multirow}
\usepackage[ngerman]{varioref}
\setcapindent{1em}
\renewcommand{\labelitemi}{--}
\usepackage{paralist}
\usepackage{pdfpages}
\usepackage{lscape}
\usepackage{float}
\usepackage{acronym}
\usepackage{eurosym}
\usepackage[babel]{csquotes}
\usepackage{longtable,lscape}
\usepackage{mathpazo}
\usepackage[flushmargin,ragged]{footmisc} % left align footnote

\urlstyle{same}  % don't use monospace font for urls

\usepackage[fleqn]{amsmath}

%adjust fontsize for part

\usepackage{sectsty}
\partfont{\large}

%Das BibTeX-Zeichen mit \BibTeX setzen:
\def\symbol#1{\char #1\relax}
\def\bsl{{\tt\symbol{'134}}}
\def\BibTeX{{\rm B\kern-.05em{\sc i\kern-.025em b}\kern-.08em
    T\kern-.1667em\lower.7ex\hbox{E}\kern-.125emX}}

\usepackage{fancyhdr}
\fancyhf{}
\pagestyle{fancyplain}
\fancyhead[R]{\thepage}

%meta
%meta

\fancyhead[L]{C. Wieder \\ %author
LIBREAS. Library Ideas, 24 (2014). % journal, issue, volume.
\href{http://nbn-resolving.de/urn:nbn:de:kobv:11-100215896}{urn:nbn:de:kobv:11-100215896}} % urn
\fancyhead[R]{\thepage} %page number
\fancyfoot[L] {\textit{Creative Commons BY 3.0}} %licence
\fancyfoot[R] {\textit{ISSN: 1860-7950}}

\title{\LARGE{\#Readwomen2014 -- Eine Online-Aktion mit nachhaltigen Auswirkungen auf unser Leseverhalten und den Literaturbetrieb?}} %title
\author{Christine Wieder} %author

\date{}
\begin{document}

\maketitle
\thispagestyle{fancyplain} 

%abstracts

%body
Als Joanne Rowling im Jahr 2013 \emph{Der Ruf des Kuckucks}
veröffentlichte, ihren ersten Kriminalroman und ihre zweite
Buchveröffentlichung nach der Harry Potter Reihe, verwendete sie ein
Pseudonym. Die Aufmerksamkeit sollte allein dem veröffentlichten Werk
gelten. Die Einrücke von Literaturkritik und Öffentlichkeit sollten
nicht durch Rowlings berühmten Namen und den damit einhergehenden
Vorurteilen überschattet werden, wie es bei ihrem zuvor veröffentlichten
Roman \emph{Ein plötzlicher Todesfall} geschehen war. Als Pseudonym
wählte Rowling den Namen Robert Galbraith. Nach eigener Aussage
beabsichtigte sie damit lediglich, den Abstand zu ihrer eigentlichen
Identität zu vergrößern.\footnote{\url{http://www.robert-galbraith.com//\#frequentlyAskedQuestions}}
Rowlings Namenswahl zog jedoch auch einige Kritik auf sich. Vielfach
warf man ihr vor, mit der Wahl eines männlichen Pseudonyms zu
Vorurteilen gegenüber weiblichem Schreiben beizutragen.\footnote{z.B.
  \url{http://www.hollywood.com/news/celebrities/55020075/jk-rowling-cuckoos-calling-robert-galbraith-pseudonym-feminist}}
Wie man Rowlings Motive auch interpretieren möchte, sie steht in einer
langen Tradition von Autorinnen, die für literarische Veröffentlichungen
anstatt ihres eigenen Namens ein männliches oder geschlechtsneutrales
Pseudonym verwenden. Emily, Anne und Charlotte Brontë veröffentlichten
ihre Romane unter den Namen Ellis, Acton und Currer Bell. Mary Ann Evans
berühmter Roman \emph{Middlemarch} wird noch heute unter ihrem Pseudonym
George Eliot verlegt. Andere Fälle sind weniger eindeutig: Harper Lee
strich ihren ersten Vornamen Nelle und publizierte unter ihrem
geschlechtsneutralen zweiten Vornamen.\footnote{\url{http://www.webdesignschoolsguide.com/library/10-famous-females-who-used-male-pen-names.html}}
Als Rowling 1997 \emph{Harry Potter und der Stein der Weisen}
veröffentlichte, tat sie das auf Anraten ihres Verlags unter dem Kürzel
\enquote{J.K.}. Es wurde befürchtet, dass ein weiblicher Vorname auf dem
Buchcover die erwünschten männlichen jungen Leser abschrecken könnte.
Angeblich seien diese nicht interessiert daran, von einer Frau verfasste
Geschichten über einen Jungen zu lesen.\footnote{\url{http://www.telegraph.co.uk/news/uknews/1349288/Harry-Potter-and-the-mystery-of-J-Ks-lost-initial.html}}
Erst nach dem großen Erfolg des ersten Harry Potter-Bandes wurde weithin
bekannt, dass es sich bei der Autorin um eine Frau handelt.~

Dies deutet darauf hin, dass es, wenigstens in der Markteinschätzung der
Verlage, nach wie vor Vorbehalte gegenüber von Frauen verfasster
Literatur gibt. Auch die Autorin und Illustratorin Joanna Walsh
beobachtete Ungleichheiten im Bereich der Literatur und nahm dies als
Anlass, das Jahr 2014 als \enquote{The year of reading women}
auszurufen. Inspiriert von anderen Menschen, die für einen begrenzten
Zeitraum ausschließlich Literatur von Frauen lasen und darüber in ihren
Blogs berichteten, gestaltete Walsh verschiedene Neujahrsgrußkarten in
der Form von Lesezeichen mit diesem Motto. Die Karten zeigen
Illustrationen von Autorinnen, darunter etwa Simone de Beauvoir und
Gertrude Stein. Auf der Rückseite der Karten sind sie mit den Namen
zahlreicher Autorinnen bedruckt, die Walsh für lesenswert hält. Walsh
zeigte Bilder dieser Grußkarten auf ihrem Blog sowie auf ihrem
Twitter-Account mit dem Hashtag \#readwomen2014 und stieß damit auf
großes Interesse.\footnote{\url{http://www.theguardian.com/lifeandstyle/womens-blog/2014/jan/20/read-women-2014-change-sexist-reading-habits}}
Der Hashtag wurde seit Anfang Januar für viele Beiträge genutzt, die
sich mit Literatur von Frauen auseinander setzten. Es werden Bücher von
Autorinnen empfohlen, Leseeindrücke beschrieben, Kritiken in
Literaturblogs verlinkt und Literaturzitate geteilt. Der größte Teil der
Beiträge ist auf Englisch, vereinzelt finden sich jedoch auch andere
Sprachen. Joanna Walsh richtete zudem einen Twitter-Account unter dem
Namen \#readwomen2014 ein, in dem sie eine Auswahl der getaggten
Beiträge retweetet.

Von Twitter ausgehend verbreitete sich \#readwomen2014 auch auf andere
Plattformen: zahlreiche Blogs und Webseiten riefen zum Lesen von
Autorinnen auf. In der Social-Reading-Plattform goodreads wurden Listen
empfehlenswerter Bücher von Autorinnen erstellt, die Walshs Empfehlungen
aufgreifen und eigene hinzufügen. Verschiedene englischsprachige Medien
berichteten über \#readwomen2014 und erzeugten so weitere
Aufmerksamkeit. Auch Bibliotheken und Buchhandlungen wurden von der
Aktion inspiriert und beteiligten sich an ihrer Verbreitung. So weist
etwa der offizielle Twitter-Account der Manchester Libraries auf
\#readwomen2014 hin.\footnote{\url{https://twitter.com/MancLibraries/status/426308874623537152}}
Walsh selbst veröffentlichte auf ihrem Blog Bilder aus Buchhandlungen,
die mit einem Verweis auf \#readwomen2014 Literatur von Frauen
bewerben.\footnote{\url{http://badaude.typepad.com/my_weblog/2014/01/readwomen2014-in-bookshops.html}}

Einige Menschen wurden von Walshs Aufruf dazu angeregt, im Jahr 2014
ausschließlich von Frauen verfasste Literatur zu lesen. Sie selbst
erklärte hingegen, dass sie 2014 auch von Männern verfasste Literatur
lesen wird. Ihr Ziel ist, jede(n) Einzelne(n) dazu anzuregen, die
eigenen Lesegewohnheiten bewusst wahrzunehmen und zu reflektieren.
Darüber hinaus sieht Walsh Leserinnen und Leser in der Verantwortung,
als Konsumentinnen und Konsumenten den Literaturbetrieb zu verändern.
Walsh geht es also nicht um das exklusive Lesen von Autorinnen, sondern
um eine verstärkte Aufmerksamkeit für von Frauen verfasste Literatur und
für die Ungleichheiten im Literaturbetrieb.\footnote{\url{http://www.theguardian.com/lifeandstyle/womens-blog/2014/jan/20/read-women-2014-change-sexist-reading-habits}}

Doch worin bestehen diese Ungleichheiten? Die meisten Leserinnen und
Leser würden vermutlich von sich selbst behaupten, dass Merkmale wie das
Geschlecht, die Herkunft oder das Alter von Autorinnen und Autoren
keinen Einfluss auf die eigene Lektüreauswahl haben und dass allein das
jeweilige literarische Werk und das jeweilige persönliches Interesse
ausschlaggebend sind. Romane von Autorinnen werden in einem ähnlichen
Umfang verlegt wie die ihrer männlichen Kollegen\footnote{Ebd.}, auch in
Buchhandlungen und Bibliotheken ist kein Mangel an von Frauen verfassten
Büchern festzustellen. Zudem haben es Autorinnen heute leichter,
Anerkennung für ihr literarisches Werk zu finden, als noch vor einigen
Jahrzehnten. Dafür spricht zum Beispiel, dass die kanadische Autorin
Alice Munro im Jahr 2013 mit dem Nobelpreis für Literatur ausgezeichnet
wurde. Damit steht das Verhältnis von Autorinnen zu Autoren in diesem
Jahrzehnt bei 1:2. In den 2000er Jahren lag es bei 3:7, in den 1990ern
ebenso. Zwischen Nelly Sachs (1966) und Nadine Gordimer (1991) wurde
dieser Preis aber sehr lange Zeit ausschließlich Männern zugesprochen.

Auch das Leseverhalten von Frauen bietet, vergleicht man es mit dem von
Männern, nicht gerade Grund zur Besorgnis. Das amerikanische
Meinungsforschungsinstitut \enquote{Pew Research Center} veröffentlichte
eine Studie zu amerikanischen Lesegewohnheiten. 82 Prozent der Frauen
hatten demnach im Vorjahr ein Buch gelesen. Dem stehen 69 Prozent der
Männer gegenüber. Zudem lasen Frauen durchschnittlich 14 Bücher pro
Jahr, männliche Leser kamen auf 10.\footnote{\url{http://www.albionpleiad.com/2014/01/readwomen2014-kicks-off-with-overwhelming-support/}}
Worauf gründet sich also \#readwomen2014?~

Das Problem besteht hauptsächlich in der Art und Weise, in der von
Frauen verfasste Literatur rezipiert und vermarktet wird. Vida, eine
Organisation amerikanischer Autorinnen, die sich mit der öffentlichen
Wahrnehmung von weiblichem Schreiben befasst, beobachtet seit mehreren
Jahren die Literaturkritiken, die in verschiedenen renommierten
Publikationen erscheinen. In Betracht gezogen wird sowohl das Geschlecht
der Autorinnen und Autoren der rezensierten Bücher als auch das der
Rezensentinnen und Rezensenten. Das Ergebnis, das die Organisation auf
ihrer Webseite veröffentlicht, ist eindeutig.~

In allen untersuchten Publikationen überwiegen Rezensionen, die von
Männern verfasst wurden. Auch die meisten rezensierten Bücher stammen
von männlichen Autoren. In der London Review of Books etwa macht der
Anteil von Frauen sowohl bei den Rezensentinnen als auch bei den
rezensierten Autorinnen deutlich weniger als 30 Prozent aus.\footnote{\url{http://www.vidaweb.org/three-years-to-stump-and-stack-and-stem}}
Diese Ergebnisse lassen den Schluss zu, dass von Frauen geschriebene
Literatur es bis heute schwerer hat, als literarisch wertvoll anerkannt
zu werden.

Rezensionen in den Rezensionszeitschriften haben zugleich einen großen
Einfluss darauf, was von der literaturinteressierten Öffentlichkeit als
lesenswert wahrgenommen wird. Bücher, die dort keine Beachtung finden,
werden auch von einem bedeutenden Teil der Leserschaft ignoriert und
damit nicht gekauft und gelesen.~

Es geht also nicht um die breite Masse an belletristischen
Veröffentlichungen, in denen Autorinnen durchaus vorkommen, sondern
darum, welcher Literatur und welcher Autorenschaft literarische Qualität
zugesprochen werden. Besprechungen von Frauen verfasster Bücher sind
nicht nur in einem geringeren Ausmaß aufzufinden, sie zeigen auch
inhaltliche Auffälligkeiten. So beobachtete Joanna Walsh, dass Werken
von Autorinnen oft Merkmale zugesprochen werden, die weniger mit dem
jeweiligen Text zu tun haben, als mit Vorurteilen über weibliches
Schreiben:

\begin{quote}
\enquote{I've listened to female writer friends grouse when their books
are given flowery covers though their writing is not; when reviews, or
even their publishers' press releases, describe their work as}delicate"
when it is forthright, \enquote{delightful} when it is satirical,
\enquote{carving a niche} when it is staking a claim.``\footnote{\url{http://www.theguardian.com/lifeandstyle/womens-blog/2014/jan/20/read-women-2014-change-sexist-reading-habits}}
\end{quote}

Auch die Vermarktung von Büchern ist häufig von Geschlechtervorurteilen
geprägt. Motive auf Buchcovern zeigen oft verniedlichende und betont
weibliche Farben und Motive, auch wenn diese in keinem Zusammenhang mit
dem Buchinhalt stehen.~

In ihren Motiven und Zielen weist \#readwomen2014 Parallelen zur
feministischen Literaturwissenschaft auf. Diese widmet sich seit
Jahrzehnten der Anerkennung und Sichtbarmachung von Autorinnen und
beschäftigt sich mit von Frauen verfassten Werken. Innerhalb ihrer
jeweiligen Disziplinen hat die feministische Literaturwissenschaft
sicherlich große Fortschritte vorangetrieben und aufgezeigt, dass die
bis heute anhaltende Marginalisierung von Frauen im Literaturbetrieb
historisch bedingt ist. In \enquote{How to Suppress Women's Writing}
beschrieb Joanna Russ 1983 die Hürden, denen sich Autorinnen gegenüber
sahen. Weibliches Schreiben und Publizieren wurde von der Umwelt kaum
akzeptiert und fand wenig Unterstützung. Zudem wurde von Frauen in
vielen Fällen erwartet, den größten Teil ihrer Zeit damit zu verbringen,
den Haushalt zu führen und die Familie zu versorgen. Folge dieser
Zustände war, dass zu der Zeit, als sich ein großer Teil des heute
akzeptierten Literaturkanons herausbildete, weniger Autorinnen als
Autoren publizierten und ihre Bücher von ihrer Umwelt in der Regel kaum
Beachtung fanden. Der Literaturkanon der Schullehrpläne besteht nach wie
vor überwiegend aus Werken, die von männlichen Autoren verfasst wurden.
Folglich wird Literatur im Sinne von \enquote{Hochliteratur} bis in die
Gegenwart vielfach mit \enquote{von männlichen Autoren produzierter
Literatur} assoziiert, ob bewusst oder unbewusst.~

Auch außerhalb des Literaturbetriebs, etwa im Bereich der Kunst, werden
kreative Erzeugnisse von Frauen oft nicht angemessen wahrgenommen. Werke
von Frauen sind nicht nur in Museen unterrepräsentiert, viele
Künstlerinnen sind zudem bei Wikipedia nicht, oder nur in Kurzeinträgen,
verzeichnet. Aus diesem Grund fanden am 1. Februar 2014 in mehr als
zwanzig Städten Europas und Nordamerikas öffentliche Zusammenkünfte von
Menschen statt, die an diesem Tag gemeinsam Wikipedia-Einträge über
kunstschaffende Frauen hinzufügten oder vervollständigten.\footnote{\url{http://artandfeminism.tumblr.com/about}}
Die verbesserte Auffindbarkeit von Künstlerinnen bei Wikipedia hat das
Potential, bei den Nutzerinnen und Nutzern von Wikipedia die Wahrnehmung
von Kunstbetrieb und Kunstgeschichte zu verändern. Insbesondere können
so junge Menschen erreicht werden, für die Wikipedia eine wichtige
Informationsquelle ist.

Wie groß ist jedoch die Wahrscheinlichkeit, dass eine Aktion wie
\#readwomen2014 nachhaltige Veränderungen im Literaturbetrieb und in der
allgemeinen Wahrnehmung bewirken kann? Die Verlage orientieren sich in
ihrer Produktgestaltung und -vermarktung vornehmlich an kommerziellen
Interessen. Einen spürbaren Einfluss von \#readwomen2014 auf diese ist
nicht anzunehmen. Das Gegenteil mag eher der Fall sein: erhöhte positive
Aufmerksamkeit für von Frauen verfasste Literatur mag somit auch solchen
Büchern zukommen, die sich in Covergestaltung und Verlagstexten
Geschlechterklischees bedienen.

\#readwomen2014 bietet vielen literaturinteressierten Menschen einen
Anlass, sich intensiver mit Literatur von Frauen auseinanderzusetzen.
Die Aktion findet zudem an Orten des Internets statt, an denen es
möglich ist, eine breite, literaturinteressierte Öffentlichkeit zu
erreichen und für die Problematik zu sensibilisieren. Auch Menschen, die
sich -- jetzt oder in Zukunft -- beruflich mit Literatur beschäftigen,
kommen so mit der Aktion in Berührung. Eine langfristige Veränderung der
Literaturkritik und des Literaturbetriebs ist deshalb vorstellbar.
Insbesondere wenn man bedenkt, dass sich die Buchbranche insgesamt in
einem Stadium der Neuorientierung befindet. Nicht nur E-Books und
Self-Publishing verändern den Literaturbetrieb grundlegend. Auch
Literaturblogs, Online-Medien und Social-Reading-Portale stehen in
Konkurrenz zum traditionellen Literaturjournalismus, der dadurch einen
Teil seiner Deutungshoheit einbüßt. Sollte es also nicht möglich sein,
die traditionellen Kanäle der Literaturkritik zu Veränderungen zu
bewegen, so scheint jetzt ein günstiger Zeitpunkt zu sein, die Relevanz
dieser Kanäle für die Zukunft zu überdenken und neue, alternative
Kommunikationsformen zu etablieren.

Wie auch immer sich \#readwomen2014 in den kommenden Monaten entwickeln
mag: die Aktion kann auch unabhängig von der Datierung weiterhin Anlass
dazu geben, sowohl die eigenen Lesevorlieben als auch das Angebot des
Literaturmarkts über die Dimension des Geschlechts hinaus zu
reflektieren. Das Anliegen, den eigenen Lesehorizont zu erweitern, lässt
sich in vielfacher Weise verfolgen. Das Lesen ermöglicht einen Einblick
in fremde Erfahrungen und Perspektiven. Eine Vielfalt der Stimmen, die
man als Lesender oder Lesende aufnimmt, kann diese Eindrücke nur noch
weiter bereichern.~

%autor

\end{document}