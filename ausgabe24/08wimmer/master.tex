\documentclass[a4paper,
fontsize=11pt,
%headings=small,
oneside,
numbers=noperiodatend,
parskip=half-,
bibliography=totoc,
final
]{scrartcl}

\usepackage{synttree}
\usepackage{graphicx}
\setkeys{Gin}{width=.6\textwidth} %default pics size

\graphicspath{{./plots/}}
\usepackage[ngerman]{babel}
%\usepackage{amsmath}
\usepackage[utf8x]{inputenc}
\usepackage [hyphens]{url}

\usepackage[colorlinks, linkcolor=black,citecolor=black, urlcolor=blue,
breaklinks= true]{hyperref}
\usepackage{booktabs} 
\usepackage[left=2.4cm,right=2.4cm,top=2.3cm,bottom=2cm,includeheadfoot]{geometry}
\usepackage{eurosym}
\usepackage{multirow}
\usepackage[ngerman]{varioref}
\setcapindent{1em}
\renewcommand{\labelitemi}{--}
\usepackage{paralist}
\usepackage{pdfpages}
\usepackage{lscape}
\usepackage{float}
\usepackage{acronym}
\usepackage{eurosym}
\usepackage[babel]{csquotes}
\usepackage{longtable,lscape}
\usepackage{mathpazo}
\usepackage[flushmargin,ragged]{footmisc} % left align footnote

\urlstyle{same}  % don't use monospace font for urls

\usepackage[fleqn]{amsmath}

%adjust fontsize for part

\usepackage{sectsty}
\partfont{\large}

%Das BibTeX-Zeichen mit \BibTeX setzen:
\def\symbol#1{\char #1\relax}
\def\bsl{{\tt\symbol{'134}}}
\def\BibTeX{{\rm B\kern-.05em{\sc i\kern-.025em b}\kern-.08em
    T\kern-.1667em\lower.7ex\hbox{E}\kern-.125emX}}

\usepackage{fancyhdr}
\fancyhf{}

\pagestyle{fancyplain}

\fancyhead[R]{\thepage}
%meta

\fancyhead[L]{U . Wimmer \\
LIBREAS. Library Ideas, 24 (2014). % journal, issue, volume.
\href{http://nbn-resolving.de/urn:nbn:de:kobv:11-100215926}{urn:nbn:de:kobv:11-100215926}} % urnLIBREAS. Library Ideas 24} %issue number
\fancyhead[R]{\thepage} %urn
\fancyfoot[L] {\textit{Creative Commons BY 3.0}} %licence
\fancyfoot[R] {\textit{ISSN: 1860-7950}}

\title{\LARGE{Ein extrem praktisches Buch} \\ \large{Rezension zu: 
Doreen Siegfried, Sebastian Johannes Nix: Nutzerbezogene Marktforschung für Bibliotheken: eine Praxiseinführung. – Berlin: De Gruyter Saur, 2014
}} %title
\author{Ulla Wimmer} %author

\date{}

\begin{document}

\maketitle
\thispagestyle{fancyplain} 





\enquote{Gibt man im Rechercheportal econbiz.de den Suchbegriff
‚Marktforschung` ein, so erhält man 19.358 Nachweise von
wissenschaftlicher Literatur zum Thema {[}\ldots{}{]}} -- So beginnt das
Vorwort zum Band \enquote{Nutzerbezogene Marktforschung für
Bibliotheken: eine Praxiseinführung}, der im Folgenden vorgestellt wird.

Was kann man auf 180 Seiten über Marktforschung schreiben, das die
anderen 19.000 Beiträge und hunderte Monographien nicht schon
ausführlich und gründlich beschrieben haben? Dies fragt nicht nur das
Vorwort, sondern vielleicht auch manche/r LeserIn. Auf diese skeptische
Frage gibt das Buch eine sehr überzeugende Antwort. Es tut nämlich -- im
Unterschied zu zahlreichen anderen Publikationen mit diesem Anspruch --
genau das, was es sagt: Es gibt eine \enquote{Praxiseinführung
{[}\ldots{}{]} für Bibliotheken}:

\enquote{Für Bibliotheken} heißt: Jedes Beispiel, jede zitierte Studie
und jede Konkretisierung stammt tatsächlich aus dem Bibliotheksbereich.
So durchgängig und konsequent bibliotheksbezogen zu denken gelingt nur
wenigen Publikationen, die dazu antreten, Methoden aus anderen
Disziplinen in die Bibliothekswelt zu \enquote{übersetzen}. Hier zeigt
sich, dass die beiden Autoren (Sebastian Nix ist Leiter der Bibliothek
des Wissenschaftszentrums für Sozialforschung (WZB) in Berlin, Doreen
Siegfried ist Leiterin der Abteilung Marketing und Public Relations der
Zentralbibliothek der Wirtschaftswissenschaften (ZBW) in Kiel) sich der
empirischen Sozialforschung über konkrete Bibliotheksprojekte genähert
haben. Die Autoren sind in ihren Beispielen und Studien um einen
ausgewogenen Bezug auf Öffentliche und Wissenschaftliche Bibliotheken
bemüht, was das Buch für beide Sparten zugänglich macht. Das ist sehr
lobenswert, wenn man bedenkt, dass die neuere Nutzerforschung in den
Öffentlichen Bibliotheken ihren Anfang genommen hat.

\enquote{Praxiseinführung} heißt: Die Auswahl und die Tiefe der
behandelten Themen orientieren sich konsequent an dem, was für ein
konkretes Marktforschungsprojekt in der Bibliothekspraxis am häufigsten
gebraucht wird und was PraktikerInnen ohne sozialwissenschaftlichen
Hintergrund als \enquote{Startkapital} benötigen. Die Stärke besteht
also nicht darin, den 19.000 Publikationen noch etwas hinzuzufügen,
sondern all das wegzulassen, was beim Einstieg in ein
bibliotheksbezogenes Forschungsprojekt noch nicht nötig ist. Diese Form
der \enquote{didaktischen Reduktion} ist nicht einfach, und sie ist hier
sehr gut gelungen. Die Autoren bringen zum Beispiel das Kunststück
fertig, auf viereinhalb Seiten nicht nur eine sehr verständliche
Darstellung der statistischen Lage- und Verteilungsmaße zu liefern,
sondern sogar noch als zusätzliches, leicht nutzbares Instrument die
Kreuztabelle einzuführen. Bei der Inferenzstatistik -- die einen
größeren Kontext- und Erklärungsaufwand erfordern würde -- beschränken
sie sich dagegen auf einen Hinweis zu deren Aussagen und
Einsatzfeldern.~

Dass eine derart kompakte Einführung nicht alle notwendigen
Informationen und Kenntnisse vermitteln kann, wird stets betont und
durch eine ausgewählte, kommentierte (!) Bibliographie ausgeglichen, die
wenige besonders empfehlenswerte Werke, Webseiten und Lernmedien
heraushebt.

Sehr erfreulich ist, dass die Autoren ihre LeserInnen nicht zu falschem
Ehrgeiz ermutigen: Man muss nicht alles selber machen, lautet die
Botschaft, gute Marktforschung bedeutet, alle vorhandenen Ressourcen zu
kennen und zu nutzen! So wird konsequent auf Möglichkeiten zur
Entlastung und Abgrenzung der eigenen Aktivitäten hingewiesen: Das sind
z.~B. vorhandene Datenquellen (nicht irgendwelche!; sondern genau die,
die im Umfeld der Hochschule oder Kommune bibliotheksrelevant sind), die
das eigene Erheben von Daten erübrigen, oder bereits durchgeführte
Studien, die manch bibliotheksrelevante Forschungsfrage (z.~B. nach dem
Informationsverhalten von Wissenschaftlern) schon geklärt haben. Das
reicht bis zu kleinen, aber (sofern man den statistischen Kontext
verstanden hat) nützlichen Helferlein, wie einem Online-Rechner zur
Berechnung der Stichprobengröße.~

Die Autoren verzichten auf eine Diskussion der verschiedenen Linien und
Differenzierungen ihres Gegenstandes: \enquote{Im vorliegenden Buch
werden Begrifflichkeiten wie ‚(bibliothekarische) Marktforschung`,
‚(Be-)Nutzerforschung` oder auch ‚Benutzungsforschung` synonym
verwendet.} (S. 4). Das ist bei einer Praxiseinführung vertretbar.
Begründen ließe sich aber durchaus, warum es von den drei Begriffen
ausgerechnet die \enquote{Marktforschung} in den Titel des Buches
geschafft hat, und nicht die anderen beiden stärker kultur- und
wissenssoziologisch orientierten Begriffe. Denn Praxisbezug heißt hier
auch: Es geht bei den Beispielen durchgängig um die Unterstützung von
Managemententscheidungen im Bibliotheksbetrieb. \enquote{Möbliere ich
den Gruppenarbeitsbereich im 2. OG neu? {[}\ldots{}{]} Investiere ich
weiterhin erhebliche Ressourcen in meine virtuelle Fachbibliothek?}
(S.~12) sind Beispiele für die Konsequenzen aus einem abgeschlossenen
Marktforschungsprojekt.~

Trotz dieses Ansatzes werden die Materialien und kompakten Darstellungen
des Buches auch für den Einstieg in Forschungsprojekte ohne
Managementbezug -- mindestens bis zum Bachelor -- hilfreich sein. Das
Buch verbindet seinen Handlungsbezug mit dem stets notwendigen, hier
sehr feinen Theorierahmen (insbesondere zu Untersuchungsdesign,
Zielsetzung und Forschungsanordnung), der die zahlreichen Praxishilfen
in einen Zusammenhang bringt.

Die Literatur und auch die Praxis der Nutzerforschung im
Bibliotheksbereich war seit ihrer Verbreitung in den 80er und 90er
Jahren vorwiegend geprägt von quantitativen Verfahren. Es ist ein großer
Verdienst dieses Buches, dass es erstmals -- und zwar in allen
Bereichen, vom Untersuchungsdesign über die Probandenauswahl und das
Interview bis hin zur Inhaltsanalyse -- gleichberechtigt sowohl in
qualitative und quantitative Verfahren einführt. Auch ethnologische
Verfahren finden dadurch Eingang in das Methodenspektrum, aus dem
PraktikerInnen zukünftig auswählen werden.

Der Aufbau des Buches orientiert sich an den Stationen eines
Marktforschungsprojektes: Das 1. Kapitel erklärt Grundbegriffe der
empirischen (Sozial-) Forschung: Ablauf eines Projektes, explorative,
deskriptive, kausale Untersuchungen, Objektivität, Reliabilität,
Validität, ethische und juristische Aspekte. Kapitel 2 führt in
unterschiedliche Forschungsdesigns und -ansätze ein: Primär- versus
Sekundärforschung, qualitative und quantitative Ansätze, Experimente,
Quer- und Längsschnitt-Untersuchungen. In Kapitel 3 geht es um die Frage
des Stichprobendesigns. Den Hauptteil des Buches nimmt das vierte
Kapitel ein, das unterschiedliche Erhebungsmethoden für nutzerbezogene
Daten beschreibt (und auf das unten näher eingegangen wird). Kapitel 5
widmet sich der Analyse und Aufbereitung der Daten bis hin zur
Gestaltung der Ergebnispräsentation.~

Mithilfe von Übungen, Checklisten und Übersichtstabellen wird konsequent
der Kontakt zur operativen Ebene gehalten: Wie wählt man -- mittels
eines \enquote{Screenings} -- Probanden für qualitative Interviews aus?
Welche Software eignet sich für eine Online-Umfrage? Welches ist das
beste Gerät zum Aufzeichnen von Interviews? Haben Sie Ihre
Ergebnispräsentation geübt? (Ja! Gute Präsentationen muss man üben!
Danke, dass das so deutlich gesagt wird.)

Das umfangreiche Kapitel 4 -- \enquote{Erhebungstechniken} beschreibt
die beiden wichtigsten Strategien zur Datengewinnung -- Befragung und
Beobachtung -- jeweils in ihren qualitativen und quantitativen
Ausprägungen.

Aufgrund ihrer weiten Verbreitung nehmen im Kapitel \enquote{Befragung}
die quantitative, standardisierte Befragung sowie -- als qualitative
Methode -- die moderierte Gruppendiskussion besonders viel Raum ein.
Nicht nur das Design und die Fragenformulierung bei einer
standardisierten Befragung, auch die Befragungsform (persönlich --
telefonisch -- schriftlich -- online) werden diskutiert.~

An der Schnittstelle zwischen Stichprobenverfahren und
Online-Befragungen findet sich allerdings eine der wenigen echten
Darstellungslücken: Für Online-Befragungen stehen heute viele
Instrumente und Tools zur Verfügung und machen die Durchführung zu einem
(vermeintlichen) Kinderspiel. Deshalb müsste hier explizit auf die
Problematik der Verzerrung von Online-Befragungsergebnissen durch
passive Selektion aufmerksam gemacht werden (also dadurch, dass der/die
ForscherIn nicht aktiv entscheidet, wer an einer Befragung teilnimmt).
Die Praxis \enquote{Wir erstellen einen Online-Fragebogen und fordern
dann auf der Uni-Homepage ‚Alle` zum Mitmachen auf} ist auch im
Bibliotheksbereich so verbreitet, dass man vor ihren methodischen
Schwächen nicht oft genug warnen kann.

Die Erhebungsform \enquote{Beobachtung} wird in ihrer Grundform kurz
vorgestellt (aber mit einem sehr überzeugenden Beispiel: der Beobachtung
von NutzerInnen am RFID-Verbuchungsauto\-maten); danach stehen drei
Sachthemen, die für Bibliotheken derzeit besonders relevant sind, im
Mittelpunkt der Methodendiskussion:~

\begin{enumerate}
\def\labelenumi{\arabic{enumi}.}
\item
  Daten zur Dienstleistungsqualität
\item
  Die Gestaltung von physischen Räumen: Daten zur Nutzung, Bewertung und
  Planung der Bibliotheksräume
\item
  Die Gestaltung von virtuellen Räumen: Daten zur Usability und
  Nutzerorientierung von elektronischen Bibliotheksangeboten.
\end{enumerate}

Zu Punkt 1 wird vor allem die Methode des Mystery Shopping --als ein
Spezialfall der Beobachtung, bei der man ohne Vorankündigung als
Durchführender der Befragung die Kundenrolle in der jeweiligen
Einrichtung einnimmt -- praktisch beschrieben. Sie wird dabei in ihren
Kontext -- dem Service-Blueprint --und im Zusammenhang mit ergänzenden
Methoden zur Dienstleistungsbewertung (z.~B. Libqual+) dargestellt.
Rechtliche und ethische Fragen beim Mystery Shopping werden ausführlich
diskutiert. Drei Beispiele aus dem Bibliotheksbereich zeigen, dass
Mystery Shopping in Bibliotheken nicht so undurchführbar ist, wie oft
geglaubt wird.

Bei Punkt 2 -- der Gestaltung von Räumen -- wird das Fototagebuch als
Spezialfall des fokussierten Interviews eingeführt. Punkt 3 -- Usability
von Web-Angeboten -- beschäftigt sich detaillierter mit Formen des
Usability Testing. Bei beiden Themen, die derzeit eine Vielzahl von
Bibliotheken beschäftigen, steht jedoch vor allem die Kombination
mehrerer Methoden im Mittelpunkt. Neben dem konsequenten Einbezug von
qualitativen Methoden ist dies der zweite große Verdienst des Buches:
dass es auf seiner sehr pragmatischen Ebene dem Methodenmix einen
breiten Raum gibt. Die sonst eher abstrakten Vorteile der
\enquote{Methoden-Triangulation} werden greifbar durch Beispielstudien,
die die Methodenkombination in verschiedenen Varianten demonstrieren.
Sie zeigen damit auch, dass ein Methodenmix im Bibliotheksbereich mit
seinen begrenzten finanziellen und personellen Mitteln durchaus möglich
ist.

Sicher ist der Band für jede der genannten Methoden nur ein Einstieg,
der durch weitere Lektüre und Beratung ergänzt werden muss. Aber er
ermöglicht es, sich in der Methodenvielfalt zu orientieren, sich anhand
der empfohlenen Literatur zielgerichtet über eine (oder mehrere)
gewählte Methode/n näher zu informieren, sich Hilfe zu suchen und die
nötigen Schritte für ein Projekt zu planen.

Eine Sorge bleibt: Hoffentlich wird dem Buch nicht gerade sein
herausragendes Plus --der enge Praxisbezug --zum Verhängnis. Denn was
die Praxis zu einem bestimmten Zeitpunkt genau trifft, veraltet
besonders schnell. Ein hochaktuelles Beispiel aus dem Jahr 2014 muss
spätestens 2016 auf seine Relevanz geprüft werden; von den zahlreichen
Deep Links, Übersichten, Beispielsammlungen usw. ganz abgesehen. Hoffen
wir, dass der Verlag in regelmäßige redaktionelle und inhaltliche
Aktualisierungen investiert und damit den stolzen Preis von 49 Euro für
180 Seiten Broschur auf Dauer rechtfertigt. Kurzfristig lohnt er sich
auf jeden Fall.~

\begin{center}\rule{3in}{0.4pt}\end{center}

\textbf{Ulla Wimmer} ist wissenschaftliche Mitarbeiterin am Institut für
Bibliotheks- und Informationswissenschaft der Humboldt-Universität zu
Berlin und unterrichtet dort u. a. im Modul Forschungsmethoden. Mit
empirischen Daten aus Bibliotheken und deren Umfeld beschäftigte sie
sich intensiv bei der Arbeit am Bibliotheksindex BIX, der Deutschen
Bibliotheksstatistik und bei der Beratung von Bibliotheken zu
Managementthemen.

\end{document}