\documentclass[a4paper,
fontsize=11pt,
%headings=small,
oneside,
numbers=noperiodatend,
parskip=half-,
bibliography=totoc,
final
]{scrartcl}

\usepackage{synttree}
\usepackage{graphicx}
\setkeys{Gin}{width=.6\textwidth} %default pics size

\graphicspath{{./plots/}}
\usepackage[ngerman]{babel}
%\usepackage{amsmath}
\usepackage[utf8x]{inputenc}
\usepackage [hyphens]{url}
\usepackage{booktabs} 
\usepackage[left=2.4cm,right=2.4cm,top=2.3cm,bottom=2cm,includeheadfoot]{geometry}
\usepackage{eurosym}
\usepackage{multirow}
\usepackage[ngerman]{varioref}
\setcapindent{1em}
\renewcommand{\labelitemi}{--}
\usepackage{paralist}
\usepackage{pdfpages}
\usepackage{lscape}
\usepackage{float}
\usepackage{acronym}
\usepackage{eurosym}
\usepackage[babel]{csquotes}
\usepackage{longtable,lscape}
\usepackage{mathpazo}
\usepackage[flushmargin,ragged]{footmisc} % left align footnote

\usepackage{listings}

\urlstyle{same}  % don't use monospace font for urls

\usepackage[fleqn]{amsmath}

%adjust fontsize for part

\usepackage{sectsty}
\partfont{\large}

%Das BibTeX-Zeichen mit \BibTeX setzen:
\def\symbol#1{\char #1\relax}
\def\bsl{{\tt\symbol{'134}}}
\def\BibTeX{{\rm B\kern-.05em{\sc i\kern-.025em b}\kern-.08em
    T\kern-.1667em\lower.7ex\hbox{E}\kern-.125emX}}

\usepackage{fancyhdr}
\fancyhf{}
\pagestyle{fancyplain}
\fancyhead[R]{\thepage}

%meta
%meta

\fancyhead[L]{K. Alexander \\ %author
LIBREAS. Library Ideas, 25 (2014). % journal, issue, volume.
\href{http://nbn-resolving.de/urn:nbn:de:kobv:11-100219239
}{urn:nbn:de:kobv:11-100219239}} % urn
\fancyhead[R]{\thepage} %page number
\fancyfoot[L] {\textit{Creative Commons BY 3.0}} %licence
\fancyfoot[R] {\textit{ISSN: 1860-7950}}

\title{\LARGE{Die Frau im Bibliothekskatalog}} %title %title
\author{Karin Alexander} %author

\setcounter{page}{8}

\usepackage[colorlinks, linkcolor=black,citecolor=black, urlcolor=blue,
breaklinks= true]{hyperref}

\date{}
\begin{document}

\maketitle
\thispagestyle{fancyplain} 

%abstracts

%body
Es gibt sie noch, die alten Band- oder Sachkataloge. In vielen
Bibliotheken stehen sie wie Denkmäler. Als Relikte einer vergangenen
Zeit bezeugen sie die gebundene (!) Vielfalt und die alphabetische oder
inhaltliche Ordnung der zugänglichen Informationen. Schon viele dieser
Informationen sind in die aktuellen Online-Kataloge der Bibliotheken
eingegangenen und stehen mit neuen technischen Mitteln zur Verfügung.
Unter der neuen Oberfläche aber stecken die alten Schemata der Ordnung
und Klassifikation. Mit den neuen technischen Möglichkeiten ist sogar
sichtbarer, wo Klassifikationen und Schlagwörter \enquote{alten},
überholten Ordnungskriterien folgen. Schließlich muss ich nicht von
einem Katalogkasten zum anderen laufen, mir nicht alles umständlich
notieren~ oder viele Bände nacheinander wälzen. Heute kann ich von einem
Begriff zum anderen klicken, bekomme sogar alle Verweisungen und die
vorhandene Literatur dazu angezeigt. Allerdings muss ich (wie immer)
wissen, was ich suche.~

Ich möchte mit diesem Beitrag am Beispiel der Schlagworte \enquote{Frau}
und \enquote{Mann} darauf aufmerksam machen, ob und in welcher Ordnung
diese Schlagworte im Katalog verwendet werden. Er soll zudem ein Beitrag
zur Diskussion sein, wie Kataloge in Zukunft besser ihr Potential
offerieren, indem die Schlagworte als intellektueller Mehrwert mehr
beachtet werden.

Bibliothekskataloge existieren nicht ohne Ordnung -- oder sie sind
unbrauchbar. Die Ordnung folgt dem Zweck, die eingegebenen Informationen
bei einer Suche auch wiederfinden zu können. In diesem Sinne ist der
Katalog ein Hilfsmittel -- für die, die ihn schaffen wie für die, die
ihn nutzen.

Die heute in Bibliothekskatalogen verwendeten Schlagwörter werden seit
2012 in der Gemeinsamen Normdatei GND geführt, in der die früher
unabhängig voneinander geführten Normdateien wie die Schlagwortnormdatei
(SWD), Personennamendatei, Gemeinsame Körperschaftsdatei und die
Einheitssachtitel-Datei des Deutschen Musikarchivs zusammengeführt
worden sind; eine separat geführte SWD gibt es nicht mehr.\footnote{Vgl.
  Umlauf, Konrad: Einführung in die die Regeln für den Schlagwortkatalog
  RSWK : mit Übungen. Berlin: Institut für Bibliotheks- und
  Informationswissenschaft der Humboldt-Universität zu Berlin 1999-
  Letzte Änderung: 03.04.2014 (Berliner Handreichungen zur Bibliotheks-
  und Informationswissenschaft. ; 66)
  \href{http://www.ib.hu-berlin.de/~kumlau/handreichungen/h66/xgr}{http://www.ib.hu-berlin.de/\textasciitilde{}kumlau/handreichungen/h66/\#gr}
  (Zugriff 14.06.2014)} Für die Recherchen zu den Beispielen in diesem
Artikel nutzte ich die GND im OPAC der Deutschen
Nationalbibliothek\footnote{\url{https://portal.dnb.de}} und die
Online-GND (OGND) auf dem Server des Bibliotheksservice- Zentrums
Baden-Württemberg.\footnote{\url{swb.bsz-bw.de/DB=2.104/}} Den Wechsel
des Bezugssystems werde ich jeweils angeben.

In der realen Welt begegnen uns Frau und Mann als zwei, der weitaus
umfangreicheren Varianz von geschlechtlichen Ausprägungen des Menschen.
Die Verschiedenartigkeit von Frau und Mann nehmen wir meist als
dichotome Zergliederung wahr. Die jahrtausendealte Entwicklung der
Menschen hat dazu geführt, dass wir diese Zweigeschlechtlichkeit nicht
nur als Non-Plus-Ultra, sondern auch noch als Wertesystem verstehen.
Ebenso funktionieren die Dichotomien von Natur/Kultur, Geist/Stoff,
Subjekt/Objekt etc. Innerhalb dieser dichotomen Denkweise ist der Mann
bekanntlich der Mensch und die Frau ihm unterlegen/untergeordnet. Daraus
ergeben sich weitere Unterschiede zwischen Männern und Frauen, die Jakob
Grimm im 19. Jahrhundert auch erstmals für die Sprache, genauer das
Geschlecht der Wörter, angewandt hat.

Die aktuellere Geschichte zeigt, dass es schon früher Versuche gab,
Klassifikationen und Normdateien geschlechtergerecht zu formulieren. So
scheiterte z.B. 1991/92 die von Dagmar Jank ins Leben gerufene
Diskussion um die \enquote{Überprüfung der Regeln für den
Schlagwortkatalog und die Schlagwortnormdatei unter dem Aspekt der
Gleichbehandlung von Frauen und Männern in der Sprache}.\footnote{Jank,
  Dagmar: Die Nicht-Gleichbehandlung von Frauen und Männern in der
  Schlagwortnormdatei : ein Offener Brief. In: Bibliotheksdienst. Berlin
  25(1991)9, S.1418-1421} Von ihren damals kritisierten Beispielen sind
die meisten auch heute nach über 20 Jahren kritikwürdig. Auf einige
ihrer konkreten Beispiele (wie Arbeitslosigkeit,
Gleichstellungsbeauftragte, misshandelter Mann u.a.) werde ich noch
genauer eingehen. Der Begriff \enquote{Mädchenhandel}, der jetzt neben
\enquote{Frau / Menschenhandel} ein Synonym für den Sachbegriff
\enquote{Frauenhandel} ist und dem Oberbegriff \enquote{Menschenhandel}
zugeordnet ist, wurde verändert. Z.B. beanstandete Jank, dass
Misshandlung nur im Zusammenhang mit Frauen, Kindern und Alten existiert
und der Aspekt, welche Person misshandelt hat, keine Rolle spielt. Der
Tataspekt ist bis heute nicht abgebildet, aber inzwischen gibt es den
Sachbegriff \enquote{Misshandelter Mann}. Interessant wäre hier zu
erfahren, von wem, warum und wann diese Änderungen erfolgten. Dieser
Prozess ist m.E. intransparent. Zwar können individuell per Formular
Korrekturanfragen an die Deutsche Nationalbibliothek (DNB) gestellt
werden, aber als Antwort erhielt ich den Verweis auf bestehende Regeln.~

Einige dieser festgelegten Regeln erscheinen mir unter den hier
verhandelten Aspekten mehr als veränderungswürdig:

\begin{enumerate}
\def\labelenumi{\arabic{enumi}.}
\itemsep1pt\parskip0pt\parsep0pt
\item
  \enquote{Für alle Sachgebiete außer Chemie und Medizin haben die
  Allgemeinnachschlagewerke Vorrang vor den
  Fachnachschlagewerken.}\footnote{Liste der fachlichen Nachschlagewerke
    für die Gemeinsame Normdatei (GND), 2014, Stand: 1. April 2014, S.
    256 (\url{http://d-nb.info/1050511964/34}; Zugriff: 22.06.2014)}
\end{enumerate}

Warum sind laut Regeln primär \enquote{Allgemeinenzyklopädien den
Fachlexika} vorzuziehen?

Ein Grund dafür könnte sein, dass sich wissenschaftliche Thesen in
allgemeinen Nachschlagewerken gewissermaßen \enquote{gesetzt} haben, sie
als Erkenntnisse und nicht als Hypothesen verhandelt werden.

Vergleichen wir unter diesem Aspekt zwei wissenschaftliche Begriffe
etwas genauer: \enquote{Higgs-Teilchen} und
\enquote{Geschlechterverhältnis}. Für beide Begriffe werden in der GND
Fachlexika herangezogen. Der erste Begriff wird in der Physik verwendet,
speziell in der Elementarteilchenphysik. Die Higgs-Bosonen wurden Ende
der 1960er Jahre hypothetisch formuliert und im Juli 2012 in einem
ersten Experiment nachgewiesen.

Dieser Begriff ist ein Sachbegriff in der GND, angegeben mit der Quelle
Fachlexikon ABC Physik (Leipzig 1986) und mit 65 Publikationen (92 bei
Titelrecherche).

Der Begriff \enquote{Geschlechterverhältnis} wird seit den frühen 1980er
Jahren (Frigga Haug), verstärkt seit den 1990er Jahren in der
BRD-Frauenforschung verwendet, die damals noch nicht
Geschlechterforschung genannt wurde.\footnote{Vgl.: Braun, Kathrin:
  Frauenforschung, Geschlechterforschung und feministische Politik. In:
  Feministische Studien. Weinheim 13(1995)2, S. 107-117} Der Begriff
existiert nicht als Sachbegriff in der GND. Er taucht als Synonym unter
dem Sachbegriff \enquote{Geschlechtsverhältnis} auf. Das ist ein
spezieller Begriff, der in der Demografie oder auch Biologie die Anteile
von Frauen bzw. Männern in einer Population angibt. Als hierarchisch
untergeordnete Sachbegriffe tauchen hier \enquote{Frauenmangel} und
\enquote{Frauenüberschuss} auf. Männeranteile werden nicht spezifiziert,
obwohl z.B. die Suche nach \enquote{Männermangel} drei Titel liefert,
gegenüber nur einem beim Wort \enquote{Frauenmangel}.

\enquote{Frauenüberschuss} liefert 8 Ergebnisse,
\enquote{Männerüberschuss} keinen Treffer.

\enquote{Geschlechterverhältnis} taucht ebenso als Synonym zum
Sachbegriff \enquote{Geschlechterbeziehung} auf, der in der GND mit
einer Quelle aus dem Lexikon Psychologie (Heidelberg: Spektrum) ohne
Jahresangabe identifiziert wird. In der Online-Version dieses
Wörterbuches gibt es aber ebenso den gesuchten Begriff
\enquote{Geschlechterverhältnisse}\footnote{Vgl.\url{http://www.spektrum.de/lexikon/psychologie/geschlechterverhaeltnisse/5795}.
  Zugriff: 20.06.2014}, sogar in der Pluralform, die der Herkunft des
Begriffs aus der US-Frauenforschung (gender relations) näher kommt.
Dieser systemische Begriff ist klar gegen den unspezifischen Begriff
\enquote{Geschlechterbeziehung} abgegrenzt\footnote{~Vgl.: Braun,
  Kathrin, a.a.O., S. 108, 115} und wurde zu einem zentralen Begriff in
der Geschlechterforschung. In der DNB bringt die Suche nach
\enquote{Geschlechterverhältnis} 1.305 Ergebnisse, nach der Pluralform
428.

Das sind im Vergleich zu \enquote{Higgs-Teilchen} deutlich mehr
Ergebnisse, die bisher nicht mit einem gesonderten Schlagwort
aufzufinden sind. Im \enquote{Lexikon zur Soziologie} von Werner
Fuchs-Heinritz, das in der Liste der fachlichen Nachschlagewerke für die
GND zitiert wird, steht der Begriff z.B. bereits in der 3., völlig neu
bearbeiteten und erweiterten Auflage (Opladen 1994)\footnote{~Fuchs-Heinritz:
  Geschlechterverhältnis. In: Ders. u .a. (Hrsg.): Lexikon zur
  Soziologie. Opladen : Westdeutscher Verlag, 1994. -- S. 235 (3.,
  völlig neu bearb. und erw. Aufl.)}. Warum ist dann
\enquote{Geschlechterverhältnisse} noch kein Sachbegriff?

\begin{enumerate}
\def\labelenumi{\arabic{enumi}.}
\setcounter{enumi}{1}
\itemsep1pt\parskip0pt\parsep0pt
\item
  Eine weitere Regel besagt: \enquote{Die weibliche Form wird verwendet,
  wenn weibliche Personengruppen Gegenstand sind; männliche und
  weibliche Form dürfen zur Bezeichnung desselben Gegenstandes nur
  verwendet werden, wenn es sich um einen Vergleich handelt \ldots{}}
\end{enumerate}

Werden also in einem Titel nur Männer als Musiker beschrieben, wird das
Schlagwort \enquote{Musiker} vergeben.

Werden nur Frauen als Ingenieurinnen analysiert, folgt der Sachbegriff
\enquote{Ingenieurin}.

Geht es um Gelehrte an der Universität Innsbruck, von denen 10 Männer
und zwei Frauen sind, wovon schon die Fotos auf dem Buchumschlag künden,
wird das Schlagwort \enquote{Gelehrter} angesetzt.\footnote{Vgl.:
  Töchterle, Karlheinz (Hrsg.): Köpfe zwischen Krise und Karriere.
  Innsbruck : Innsbruck Univ. Press, 2010. - 96 S.} Als Antwort auf
meine Korrekturanfrage erhielt ich von der DNB den Verweis auf die o.g.
Regel. Der Begriff \enquote{Gelehrter} sei dabei geschlechtsunabhängig
und die entsprechende Ansetzung der Schlagwörter im Hause so
beschlossen. Das Schlagwort \enquote{Weibliche Gelehrte} sei seit 1994
nur 12 Mal vergeben worden. Bei den entsprechenden Veröffentlichungen
handele es sich ausdrücklich um solche zu weiblichen Gelehrten als
durchgängiges Thema.

Abgesehen davon, dass auch eine Porträtsammlung zur wissenschaftlichen
Karriere von Frauen und Männern an einer Universität einen Vergleich
dieser Karrieren impliziert, und damit auch das Schlagwort
\enquote{Weibliche Gelehrte} rechtfertigen würde, war es nicht die
Absicht dieses Werkes, weshalb die o.g. Regel gilt. Diese Regelanwendung
und -auslegung widerspricht aber:

\begin{itemize}
\item
  \begin{enumerate}
  \def\labelenumi{\alph{enumi})}
  \itemsep1pt\parskip0pt\parsep0pt
  \item
    dem Hauptzweck von Verschlagwortung, nämlich das im Werk Enthaltene
    wiederauffindbar zu machen. \enquote{Die RSWK sind ein Regelwerk für
    die \textbf{intellektuelle} Beschlagwortung. \textbf{Ausschlaggebend
    ist der Inhalt} eines Werkes, nicht die Titelformulierung.} Demnach
    müsste \enquote{Gelehrter} und \enquote{Weibliche Gelehrte} vergeben
    werden.\footnote{Vgl. Umlauf, Konrad. a.a.O.}
  \end{enumerate}
\item
  \begin{enumerate}
  \def\labelenumi{\alph{enumi})}
  \setcounter{enumi}{1}
  \itemsep1pt\parskip0pt\parsep0pt
  \item
    Diese Regel widerspricht dem Interesse von Suchenden, denn sie
    möchten so schnell, so effektiv und so viel wie mögliche konkrete
    Ergebnisse zu ihren Anfragen finden. Die zwei Wissenschaftlerinnen
    des erwähnten Buches wären aber nicht auf ihrer Ergebnisliste der 12
    Titel.
  \end{enumerate}
\item
  \begin{enumerate}
  \def\labelenumi{\alph{enumi})}
  \setcounter{enumi}{2}
  \itemsep1pt\parskip0pt\parsep0pt
  \item
    Folglich widerspricht diese Regel auch jeglicher Ökonomie. Warum
    wird das Buch verschlagwortet, ohne seinen Nutzwert, d.h. den Inhalt
    des Buches, voll anzugeben? Zwei gelehrte Frauen werden nicht
    erwähnt! Alle, die zum Thema \enquote{Frauen an Universitäten}
    forschen, sind gezwungen, alle Titel unter \enquote{Gelehrter}
    durchzuforsten; in diesem Fall wären das laut DNB 3.608 Titel! In
    dieser Menge stecken garantiert mehr Werke zu weiblichen Gelehrten
    als die 12 Titel!
  \end{enumerate}
\end{itemize}

Diese Beispiele zeigen, dass geschlechtergerechte Formulierungen in den
Normdateien eine besondere Herausforderung, weil bisher nicht
bearbeitetes Feld, in der übergeordneten Problematik von
Kataloganreicherungen und verbesserter inhaltlicher Erschließung sind.
Wenn nur ca. 30 bis 50\% aller Titel überhaupt inhaltlich
verschlagwortet werden, dann wird klar, wie viele Potenzen hier noch
nicht ausgeschöpft sind.\footnote{Vgl. z.B. Wiesenmüller, Heidrun:
  Sacherschließungsdaten in Bibliothekskatalogen : gestern, heute,
  morgen ;~Vortrag auf der VDB-Fortbildung \enquote{Gegenwart und
  Zukunft der Sacherschließung} am 6.10.2011 in
  Leipzig.\url{http://de.slideshare.net/heidrunw/wiesenmueller-sacherschliessung-in-bibliothekskatalogen}.
  Zugriff: 20.06.2014}

Auffallend an der Regel zur Verwendung weiblicher Formen für
Schlagwörter in der GND ist, dass sie genau dem Prinzip des generischen
Maskulinums folgt. Auch danach sind, wie Senta Trömel-Plötz schon 1980
schrieb \enquote{99 Lehrerinnen und ein Lehrer =100 Lehrer}!\footnote{Trömel-Plötz,
  Senta: Frauensprache : Sprache der Veränderung. Frankfurt/M. :
  Fischer, 1990. -- S. 95 (Die Frau in der Gesellschaft ; 3725)}
Schlagworte sind eben Worte und unterliegen damit den Regeln von
Rechtschreibung und Grammatik, aber auch dem real stattfindenden
Sprachwandel. Da es sich im Beispiel \enquote{Gelehrte vs.~Weibliche
Gelehrte} um Berufsbezeichnungen handelt, müsste hier auch die
Überlegung der Duden-Redaktion von 1998 beachtet werden. Danach sollten
Berufsbezeichnungen besser dem Geschlecht der Personen entsprechend
verwendet werden\footnote{Der Grammatik-Duden von 1998 betont:
  \enquote{Besonders bei Berufsbezeichnungen und Substantiven, die den
  Träger bzw. die Trägerin eines Geschehens bezeichnen (Nomina Agentis),
  wird die Verwendung des generischen Maskulinums immer mehr abgelehnt.}
  Vgl.: Dudenredaktion (Hrsg.): Duden : Grammatik der deutschen
  Gegenwartssprache. 6. neu bearb. Aufl. Mannheim ; Leipzig u.a. :
  Dudenverlag, 1998, S. 200 {[}Der Duden ; 4{]}}, also
\enquote{Gelehrter} und \enquote{Weibliche Gelehrte}.

Welches Bild von Frauen und Geschlechterverhältnissen steckt in der
Struktur der GND mit all ihren Sachbegriffen, Oberbegriffen,
thematischen Bezügen, hierarchischen Unterbegriffen?

Dazu hier nur drei ausgewählte Beispiele zur Diskussion:

\begin{enumerate}
\def\labelenumi{\arabic{enumi}.}
\itemsep1pt\parskip0pt\parsep0pt
\item
  Die angesetzten Sachbegriffe sind quantitativ und qualitativ
  uneinheitlich in der Ansetzung.
\end{enumerate}

Schon in der alten SWD standen quantitativ mehr Begriffe beim
Sachbegriff \enquote{Frau} als bei \enquote{Mann}. Ob das schon allein
beweist, dass der Mann gleich der Mensch ist, also das Allgemeine und
die Frau das Abweichende, Untergeordnete, Spezielle und deshalb mehr
Begriffe zur Beschreibung braucht, sei dahingestellt. Die bei den
Sachbegriffen \enquote{Frau} und \enquote{Mann} verwendeten
Unterbegriffe unterscheiden sich v.a. auch qualitativ. Außer
\enquote{Lebemann} und \enquote{Macho} gab es bei \enquote{Mann} keine
weiteren Begriffe mit negativen oder besonderen Konnotationen. Bei
\enquote{Frau} verwiesen viele Unterbegriffe auf Aspekte von
Behinderung/Versehrtheit (Blinde Frau, Taubstumme Frau, Weibliche Tote),
auf Opferpositionen (Misshandelte Frau, Getrenntlebende Frau) oder auf
negative Konnotationen wie Sünderin und Weibliche Radikale.\footnote{~Vgl.
  Aleksander, Karin: Gendern heißt ändern! : Erfahrungen aus der
  Geschichte der Genderbibliothek des ZtG an der Humboldt-Universität zu
  Berlin. In: Niedermair, Klaus (Hrsg.): Die neue Bibliothek - Anspruch
  und Wirklichkeit : 31. Österreichischer Bibliothekartag ; Innsbruck
  18. - 21. Oktober 2011. - Graz {[}u.a.{]} : Neugebauer, 2012. -- S.
  335}

Auch in der OGND fällt wieder die unterschiedliche Quantität auf: 265
Sachbegriffe unter \enquote{Frau} gegenüber 57 für \enquote{Mann}. Die
Liste für \enquote{Frau} ist aufgebläht durch alle möglichen
Länderzuge\-hörigkeiten von Frauen, die bei \enquote{Mann} in der Liste
der 57 fehlen, obwohl sie z.T. auch als Sachbegriffe in der OGND
vorhanden sind. So findet sich von der Bosnierin über die Marokkanerin
bis zur Zyprerin auch eine Keltin oder Kroatin, wobei es keinen Kelten
und Kroaten gibt, weil hier die Regel angewandt werden muss, dass
Bezeichnungen für Personen- und Ländergruppen im Plural anzusetzen sind.
Deshalb gibt es eine \enquote{Hugenottin}, aber nur
\enquote{Hugenotten}. Die männliche Pluralform steht also auch hier für
die allgemeine Gruppe, unabhängig vom einzelnen Geschlecht.

\begin{enumerate}
\def\labelenumi{\arabic{enumi}.}
\setcounter{enumi}{1}
\itemsep1pt\parskip0pt\parsep0pt
\item
  Es gibt Sachbegriffe, die nur für Männer bzw. nur für Frauen
  angewendet werden, obwohl sie für beide Geschlechter zutreffend
  formuliert werden müssten -- und das einheitlich.
\end{enumerate}

An den folgenden Beispielen zeigt sich, wie unterschwellig schon der
unter 3. zu besprechende Aspekt enthalten ist, dass nämlich stereotype
Geschlechtsrollenmodelle die unbewusste Basis der Ansetzungen bilden.
Hier sollen deshalb Beispiele gezeigt werden, die auf einer mehr
sichtbaren Ebene zeigen, dass der \enquote{Mann} das Allgemeine
verkörpert und die \enquote{Frau} das Abgeleitete. Das zeigt sich v.a.
daran, welchem Oberbegriff bzw. Synonymen oder thematischen Bezügen ein
Sachbegriff zugewiesen wurde.

Zum Beispiel gibt es den Sachbegriff \enquote{First Lady}, definiert als
\enquote{Ehefrau von Staatschefs u. Ministerpräsidenten}.
Dementsprechend lauten die Synonyme \enquote{Staatsoberhaupt / Ehefrau}
und \enquote{Präsident / Ehefrau}. Die männliche Entsprechung fehlt
(noch).

Ähnlich gibt es \enquote{Schönheitswettbewerb} und \enquote{Misswahl}
nur für Frauen, ohne männliche Entsprechungen. Zum Begriff
\enquote{Photomodell} wird sowohl auf die verwandten Begriffe
\enquote{Mannequin} als auch \enquote{Dressman} verwiesen; in der
Online-GND auch unter dem Sachbegriff \enquote{Model
\textless{}Beruf\textgreater{}}. Dort erscheinen zwei Titel zu
\enquote{Dressman}. Gebe ich das Wort in die Suchmaske (als Stichwort)
ein, erscheinen 99 Titel, von denen 97 über weibliche Models handeln und
die zwei Titel zu Dressman.

Für den Sachbegriff \enquote{Hose} gibt es zahlreiche weitere
Sachbegriffe wie \enquote{Jeans} und \enquote{Knickerbocker} und auch
\enquote{Damenhose}, aber keine Herren- oder Männerhose. Die zu
\enquote{Hose} zugeordneten Titel enthalten trotzdem Bezüge zu
Damenhosen, Frauen in Hosen etc., weil zusätzlich zu \enquote{Hose} auch
mit \enquote{Frauenkleidung} verschlagwortet wurde. Suche ich nach
\enquote{Frauenkleidung und Hose} erhalte ich drei Titel. Bei der Suche
nach \enquote{Männerkleidung und Hose} erhalte ich keinen Treffer; auch
nicht bei Herrenkleidung. Bei \enquote{Mann und Hose} gibt es drei
Ergebnisse, bei \enquote{Frau und Hose} 23, aber ohne den wichtigen
Titel von Gundula Wolters, der bei \enquote{Frauenkleidung und Hose}
erschien!

Die Beispiele zeigen, dass es Suchenden schwer gemacht wird, die
richtigen Ergebnisse für ihre Suchen zu finden, weil die Titel
uneinheitlich verschlagwortet werden.

Ebenso uneinheitlich ist die Geschlechterzuordnung beim Sachbegriff
\enquote{Arbeitslosigkeit}. Auch hier gibt es
\enquote{Frauenarbeitslosigkeit} aber keine Männerarbeitslosigkeit,
dafür den Sachbegriff \enquote{Arbeitsloser}, mit den verwandten
Sachbegriffen \enquote{Arbeitslosigkeit} und \enquote{Arbeitslose Frau};
Keine \enquote{Arbeitslose} oder \enquote{Arbeitsloser Mann}, dafür
viele speziellere Begriffe.

Ein Beispiel für die Zuordnung zu einem allgemein als weiblich
konnotierten Thema ist der Begriff \enquote{Männliche Prostitution}. Er
wird dem allgemeinen Oberbegriff \enquote{Prostitution} untergeordnet.
Als Synonym taucht hier die Verweisung \enquote{Mann / Prostitution}
auf. Demgegenüber gibt es \enquote{Weibliche Prostitution} nur als
Synonym unter \enquote{Prostitution}. Der Sachbegriff
\enquote{Prostitution} selbst ist dem Oberbegriff
\enquote{Sozialverhalten} zugeordnet und bringt keinen Bezug zu
männlicher Prostitution:\footnote{~Vgl.:
  Online-GND.\url{http://swb.bsz-bw.de/DB=2.104/SET=17/TTL=1/CLK?IKT=12\&TRM=209073810\&NOABS=Y\&REC=*}}

\begin{verbatim}
                PPN: 209073810
         GND-Nummer: 4047516-5     http://d-nb.info/gnd/4047516-5Link
                     zu diesem Datensatz in der GND
      Alte Norm-Nr.: 4047516-5 ( in der "swd" vor der GND-Migration)
  Frühere Ansetzung: in swd:|s|Prostitution
             Quelle: M
     GND-Systematik: 9.3d [Sozialisation, Sozialverhalten]
       DDC-Notation: 176.5 ; 306.74 ; 331.76130674 ; 338.4730674 ;
                     363.44 ; 364.1534
            Synonym: Gewerbliche Unzucht
                     Weibliche Prostitution
                     Sexarbeit
        Oberbegriff: Sexualverhalten [Oberbegriff allgemein]
 Thematischer Bezug: Bordell [Verwandter Begriff, allgemein]
\end{verbatim}

Daneben gibt es Sachbegriffe wie \enquote{Gleichstellungsbeauftragte}
als Synonym für \enquote{Frauenbeauftragte} weiterhin bisher nur für
Frauen. Es scheint, als ob die Anforderungen des Gender Mainstreaming
bisher v.a. in die Richtung gingen, fehlende Sachbegriffe für Frauen
\enquote{nachzuholen} (z.B.: Weibliche Zwangsrekrutierte,
\ldots{}Drogenabhängige, \ldots{}Vermisste, \ldots{}Strafentlassene,
Weibliches Publikum, \ldots{} Parteimitglied etc.). Die hier jeweils
verwendeten Substantive tauchen beim Schlagwort \enquote{Mann} gar nicht
auf. Es gibt aber bereits den Sachbegriff \enquote{Entbindungspfleger}
unter demselben Oberbegriff \enquote{Medizinisches Personal} wie
\enquote{Hebamme}!

\begin{enumerate}
\def\labelenumi{\arabic{enumi}.}
\setcounter{enumi}{2}
\itemsep1pt\parskip0pt\parsep0pt
\item
  Bei vielen Sachbegriffen basiert die Zuordnung auf unbewussten
  Geschlechterstereotypen, die dringend aufgelöst werden müssen.
\end{enumerate}

Diese Unterordnung von Frauenaspekten unter Männerallgemeinheiten ist am
schwersten zu durchschauen und deshalb auch nur mit wachsender
Erkenntnis, Überzeugung und Voranschreiten der gesellschaftlichen
Entwicklung zu verändern. Um diese geschlechterungerechten Zuordnungen
zu erkennen, braucht es den geschlechtersensiblen Blick, von dem der
Soziologe Pierre Bourdieu schrieb:

\begin{quote}
``Wenn es darum geht, die soziale Welt zu denken, kann man die
Schwierigkeiten bzw. Risiken gar nicht hoch genug veranschlagen. Die
Macht des Präkonstruierten liegt darin, daß es zugleich in die Dinge und
in die Köpfe eingegangen ist und sich deshalb mit einer Scheinevidenz
präsentiert, die unbemerkt durchgeht, weil sie selbstverständlich ist.
Der Bruch ist eigentlich eine Konversion des Blicks, und vom Unterricht
in soziologischer Forschung kann man sagen, daß er zuallererst lehren
muß, \enquote{mit anderen Augen zu sehen} \ldots{} Und das ist nicht
möglich ohne eine echte Konversion, eine metanoia, eine mentale
Revolution, einen Wandel der ganzen Sicht der sozialen Welt.
\end{quote}

\begin{quote}
Was man den \enquote{epistemologischen Bruch} nennt, also die
vorübergehende Außerkraftsetzung der gewöhnlichen Präkonstruktionen und
der gewöhnlich bei der Realisierung dieser Konstruktionen angewandten
Prinzipien, setzt oft einen Bruch mit den Denkweisen, Begriffen,
Methoden voraus, die allen Anschein des common sense, der gewöhnlichen
Alltags- und Wissenschaftsvernunft (also alles dessen, was die
herrschende positivistische Disposition honoriert und anerkennt) für
sich haben.``\footnote{Bourdieu, Pierre: Reflexive Anthropologie /
  Pierre Bourdieu und Loïc J. D. Wacquant. - Frankfurt/M.: Suhrkamp,
  1996. - S. 284f.}
\end{quote}

\begin{quote}
\enquote{Ob wir wollen oder nicht, der Mann oder die Frau, welche die
Analyse durchführen, sind selbst Teil des Objekts, das sie zu begreifen
versuchen. Denn er oder sie hat in Gestalt unbewußter Schemata der
Wahrnehmung und der Anerkennung die historisch sozialen Strukturen
männlicher Herrschaft internalisiert. Unser erstes Gebot muß deshalb
sein, eine praktische Strategie zu finden, die uns zur methodischen
Objektivierung des Subjekts wissenschaftlicher Objektivierung befähigt:
einen Kunstgriff zur Aufdeckung der Strukturen des archaischen
Unbewußten, das wir unserer Ontogenese und Phylogenese als
geschlechtliche Wesen verdanken und das dazu führt, daß wir an eben dem
Phänomen teilhaben, das wir ergründen wollen.}\footnote{Bourdieu,
  Pierre: Männliche Herrschaft revisited. In: Feministische Studien.
  Weinheim 15(1997)2. S. 88f.}
\end{quote}

Ein Begriff dieser Kategorie ist z.B. \enquote{Gleichstellungspolitik},
der dem Oberbegriff \enquote{Frauenpolitik} zugeordnet ist und letzterer
zu den verwandten Begriffen \enquote{Frauenbewegung} und
\enquote{Gleichberechtigung}. Erst beim Begriff
\enquote{Gleichberechtigung} gibt es mit den Synonymen \enquote{Mann /
Frau / Gleichberechtigung} und \enquote{Mann / Frau / Gleichstellung}
einen Bezug zu Männern. Die \enquote{Gleichstellungspolitik} ist der
\enquote{Frauenpolitik} untergeordnet und diese wiederum dem Oberbegriff
\enquote{Sozialpolitik}. Einen Sachbegriff \enquote{Männerpolitik} gibt
es nicht, ebenso wenig \enquote{Geschlechterpolitik}!

An diesem Beispiel zeigt sich deutlich die alte Denkweise einer
gesondert existierenden \enquote{Frauenfrage}, die mit Frauenpolitik
gelöst werden muss.

Ein anderes Beispiel ist der Sachbegriff \enquote{Misshandelter Mann},
der nach der Kritik von Dagmar Jank damals noch gänzlich fehlte, der
aber nach 2010 aufgenommen worden sein muss.\footnote{In einem Artikel
  zu Frauenbibliotheken, der seit 2010 im Bibliotheksportal erscheint,
  hatte ich dieses Beispiel noch selbst angeführt, was zu korrigieren
  ist.
  Vgl.:\url{http://www.bibliotheksportal.de/bibliotheken/bibliotheken-in-deutschland/bibliothekslandschaft/frauenbibliotheken.html}.
  Zugriff: 23.06.2014} Es existiert auch der Sachbegriff
\enquote{Misshandelte Frau}, aber beide werden völlig unterschiedlichen
Oberbegriffen zugeordnet:

\begin{longtable}[c]{@{}ll@{}}
\toprule\addlinespace
\begin{minipage}[t]{0.47\columnwidth}\raggedright
\textbf{Sachbegriff}(OGND)
\end{minipage} & \begin{minipage}[t]{0.47\columnwidth}\raggedright
\textbf{Oberbegriff}
\end{minipage}
\\\addlinespace
\begin{minipage}[t]{0.47\columnwidth}\raggedright
Misshandelter Mann
\end{minipage} & \begin{minipage}[t]{0.47\columnwidth}\raggedright
\begin{itemize}
\itemsep1pt\parskip0pt\parsep0pt
\item
  Mann~
\end{itemize}
\end{minipage}
\\\addlinespace
\begin{minipage}[t]{0.47\columnwidth}\raggedright
\end{minipage} & \begin{minipage}[t]{0.47\columnwidth}\raggedright
\begin{itemize}
\itemsep1pt\parskip0pt\parsep0pt
\item
  Misshandlung
\end{itemize}
\end{minipage}
\\\addlinespace
\begin{minipage}[t]{0.47\columnwidth}\raggedright
Misshandelte Frau
\end{minipage} & \begin{minipage}[t]{0.47\columnwidth}\raggedright
Frau
\end{minipage}
\\\addlinespace
\bottomrule
\end{longtable}

Warum fehlt bei \enquote{Misshandelte Frau} die Zuordnung zum
Oberbegriff \enquote{Misshandlung}? Letzterer offeriert dann als
hierarchisch untergeordnet Sachbegriffe:

Misshandelter Mann / Häusliche Gewalt / Altenmisshandlung /
Kindesmisshandlung.

Was wird hier warum nicht zugeordnet?

Überhaupt ist die Suche nach Themen wie \enquote{Gewalt gegen Frauen
oder Männer} sehr schwer, weil die Erläuterungen besagen: für Gewalt
gegen Frauen bzw. Männer in allen Bereichen verknüpfe die Schlagwörter
Gewalt und Mann bzw. Gewalt und Frau oder laut OGND Gewalt / Frau etc.
Dieses Vorgehen ist unsystematisch. Es deckt v.a. nicht den Tataspekt
auf. Bei Eingabe der entsprechenden Suchbegriffe tauchen sowohl bei
\enquote{Gewalt Mann} als auch bei \enquote{Gewalt Frau} Titel zum Thema
Gewalt gegen Frauen auf. Auch hier wird durch ein unkonkretes Schlagwort
sehr viel Rechercheaufwand abgefordert, wenn konkret nach Gewalt gegen
Frauen oder Gewalt gegen Männer gesucht wird.

Ähnliches Potential wird verschenkt, wenn ein Titel wie
\enquote{Rechtsextremismus und Gender} mit den Schlagworten
\enquote{Rechtsradikalismus} und \enquote{Geschlechterforschung}
verschlagwortet wird. Was auf den ersten Blick positiv erscheint, weil
der Begriff \enquote{Geschlechterforschung} vergeben wurde, ist auf den
zweiten Blick unkonkret und geht weit über den Titel hinaus (was laut
Regeln auch nicht statthaft ist). In dem Sammelband finden sich von 16
Titeln 8, die sich mit Rechtsextremismus und Männern beschäftigen und 7,
die Frauen und Mädchen in diesem Feld untersuchen. Bei einer Recherche
zu \enquote{Rechtsradikalismus} und Frau oder Mann, wird dieser
Sammelband aber nicht angezeigt.

\section*{Zusammenfassung}\label{zusammenfassung}

Wenn über die Qualität von Katalogen diskutiert wird, sollte das
Potential, das eine gendersensible Verschlagwortung bietet, genutzt
werden. Dazu brauchen wir mehr Analysen, die Lücken, Fallen und Fehler
in den bisherigen Systematiken und Klassifikationen auf der Grundlage
der Ergebnisse der Geschlechterforschung aufzeigen. Die existierenden
Frauen-, Lesben- und Genderbibliotheken im deutschsprachigen Dachverband
i.d.a.\footnote{Vgl. \url{http://www.ida-dachverband.de/home/}} haben
dazu eine über Jahrzehnte ausgebildete Expertise angesammelt und sind
bereit zur Kooperation.

%autor

\end{document}