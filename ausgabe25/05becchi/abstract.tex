\begin{center}
\sffamily{\textbf{ABSTRACT}}
\end{center}

\begin{abstract}
\small
Nach zehn Exiljahren in London kam die jüdische Journalistin Jella
Lepman 1946 im Auftrag der amerikanischen Regierung zurück nach
Deutschland, als \enquote{Beraterin für die kulturellen und erzieherischen
Belange der Frauen und Kinder}. Um der geistigen Verarmung der deutschen
Nachkriegskinder entgegenzuwirken, organisierte sie eine große
Internationale Jugendbuchausstellung, die im ganzen Land gezeigt wurde
und später den Grundbestand der Internationalen Jugendbibliothek in
München bildete. Aus Amerika führte Jella Lepman ein fortschrittliches
Konzept für die Gestaltung und Leitung einer Jugendbibliothek ein, das
zunächst auf viel Widerstand von Seiten der ausgebildeten deutschen
Bibliothekare stieß. Die von ihr gegründete Bibliothek ist heute
weltweit die bedeutendste Institution dieser Art. Der Beitrag
porträtiert diese außergewöhnliche Frau, die keine ausgebildete
Bibliothekarin war und doch das Bibliothekswesen im Jugendbereich in
Deutschland revolutionierte, den deutschen Kinder- und Jugendbuchmarkt
zu einem der internationalsten überhaupt gemacht hat und ihr ganzes
Leben der Verbreitung hochwertiger Kinder- und Jugendliteratur als
Beitrag zur Völkerverständigung widmete.

\end{abstract}

\begin{abstract}
\small

After having spent ten years in exile in London, the Jewish journalist
Jella Lepman came back to Germany in 1946 on behalf of the American
government as \enquote{Special Adviser for Women’s and Youth affairs}. In order
to countervail the lack of imagination of the children in post-war
Germany, she organized a great exhibition of the best children’s books
from various nations. The exhibition travelled through the whole country
and afterwards constituted the core of the International Youth Library
in Munich. From the USA Jella Lepman imported a progressive concept for
the organization and administration of a youth library, which initially
met the resistance of the professional German librarians. The library
she founded is now the most important institution of this type in the
world. This article portrays this extraordinary woman who wasn’t a
skilled librarian and yet revolutionized the youth library system in
Germany making the German market of juvenile literature one of the most
international ones. In order to promote the understanding among nations,
she dedicated herself to the cause of spreading high quality children’s
books. \end{abstract}

\begin{center}\rule{3in}{0.4pt}\end{center}