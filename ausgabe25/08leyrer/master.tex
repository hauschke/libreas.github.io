\documentclass[a4paper,
fontsize=11pt,
%headings=small,
oneside,
numbers=noperiodatend,
parskip=half-,
bibliography=totoc,
final
]{scrartcl}

\usepackage{synttree}
\usepackage{graphicx}
\setkeys{Gin}{width=.6\textwidth} %default pics size

\graphicspath{{./plots/}}
\usepackage[ngerman]{babel}
%\usepackage{amsmath}
\usepackage[utf8x]{inputenc}
\usepackage [hyphens]{url}
\usepackage{booktabs} 
\usepackage[left=2.4cm,right=2.4cm,top=2.3cm,bottom=2cm,includeheadfoot]{geometry}
\usepackage{eurosym}
\usepackage{multirow}
\usepackage[ngerman]{varioref}
\setcapindent{1em}
\renewcommand{\labelitemi}{--}
\usepackage{paralist}
\usepackage{pdfpages}
\usepackage{lscape}
\usepackage{float}
\usepackage{acronym}
\usepackage{eurosym}
\usepackage[babel]{csquotes}
\usepackage{longtable,lscape}
\usepackage{mathpazo}
\usepackage[flushmargin,ragged]{footmisc} % left align footnote

\urlstyle{same}  % don't use monospace font for urls

\usepackage[fleqn]{amsmath}

%adjust fontsize for part

\usepackage{sectsty}
\partfont{\large}

%Das BibTeX-Zeichen mit \BibTeX setzen:
\def\symbol#1{\char #1\relax}
\def\bsl{{\tt\symbol{'134}}}
\def\BibTeX{{\rm B\kern-.05em{\sc i\kern-.025em b}\kern-.08em
    T\kern-.1667em\lower.7ex\hbox{E}\kern-.125emX}}

\usepackage{fancyhdr}
\fancyhf{}
\pagestyle{fancyplain}
\fancyhead[R]{\thepage}

%meta
%meta

\fancyhead[L]{K. Leyrer \\ %author
LIBREAS. Library Ideas, 25 (2014). % journal, issue, volume.
\href{http://nbn-resolving.de/urn:nbn:de:kobv:11-100219295
}{urn:nbn:de:kobv:11-100219295}} % urn
\fancyhead[R]{\thepage} %page number
\fancyfoot[L] {\textit{Creative Commons BY 3.0}} %licence
\fancyfoot[R] {\textit{ISSN: 1860-7950}}

\title{\LARGE{Das Geschlecht spukt in der Stadtbibliothek: Ein Aufruf für genderneutrale Bibliotheksangebote}} %title %title
\author{Katharina Leyrer} %author

\setcounter{page}{76}

\usepackage[colorlinks, linkcolor=black,citecolor=black, urlcolor=blue,
breaklinks= true]{hyperref}

\date{}
\begin{document}

\maketitle
\thispagestyle{fancyplain} 

%abstracts

%body
Die Leiterin der Stadtbibliothek in einer bayerischen Kleinstadt führt
durch die Kinder- und Jugendbücherei: \enquote{Und hier sind die Bücher
für die Erstleser: da die Fußballbücher für die Jungs und dort die
Pferdebücher für die Mädchen}. In der Belletristik-Abteilung der Jenaer
Stadtbibliothek reihen sich in einem Regal mit der Beschriftung
\enquote{Familie / Frauen / Liebe} Romane mit Titeln wie \enquote{In
einer heißen Sommernacht}, \enquote{Was dem Herzen gefällt} und
\enquote{Strom des Schicksals}\footnote{In einer heißen Sommernacht :
  Roman / Sandra Brown.~ Aus dem Amerikan. von Claudia Geng. - Augsburg
  : Weltbild, 2012. - 287 S.Was dem Herzen gefällt / Ilse Gräfin von
  Bredow. - München {[}u.a.{]} : Scherz, 2007. - 255 S.; Strom des
  Schicksals : Roman / Gwen Bristow. - Augsburg: Weltbild, 479 S.}
aneinander.

Das Thema \emph{Gender} spielt im deutschen Bibliothekswesen momentan
kaum eine Rolle: Vielmehr stoßen Geschlechterfragen in Bezug auf
Bibliotheken auf Unverständnis oder gar Ablehnung. Dabei hat der Umgang
mit Geschlechterrollen täglich Auswirkungen auf die bibliothekarische
Arbeit:~ Auf das Angebot der Bibliothek, auf dessen Präsentation und
damit auch auf die Benutzer*innen.

\section*{Welche Rolle spielt das Thema \emph{Gender} in
Bibliotheken
bislang?}\label{welche-rolle-spielt-das-thema-gender-in-bibliotheken-bislang}

Im Jahr 1999 legte ein Kabinettsbeschluss der Bundesregierung mit der
Strategie des \emph{Gender Mainstreaming} fest, dass die Gleichstellung
zwischen Männern und Frauen als Querschnittsaufgabe gefördert werden
muss. Staatliches Handeln muss daher auf allen Ebenen und in allen
Bereichen ständig auf seine geschlechtsspezifischen Auswirkungen
überprüft werden, um die Benachteiligung von Frauen und Männern zu
beseitigen. Als größtenteils öffentliche Einrichtungen müssen auch
Bibliotheken diese Richtlinie umsetzen.

Seither sind 15 Jahre vergangen, doch ein Blick aus der
Gender-Perspektive auf das Bibliothekswesen erfasst ein trauriges Bild:
Die ekz bietet Aufkleber für die Interessenskreise \enquote{Männer} und
\enquote{Frauen} an;~ in Kinder- und Jugendbibliotheken gibt es jeweils
Regale für Jungen und Mädchen; Bibliothekar*innen raten Jungen
eindringlich davon ab, ein Buch, in dem es um eine Prinzessin geht, zu
lesen. Angebote in Bibliotheken richten sich speziell an Angehörige
eines Geschlechts -- ordnen damit auch bestimmte Inhalte dem jeweiligen
Geschlecht zu. Bibliothekarisches Handeln wirkt also nicht -- wie es das
Konzept \emph{Gender Mainstreaming} fordert -- auf die Beseitigung
geschlechtsspezifischer Benachteiligung hin, sondern trägt im Gegenteil
zur Affirmation der bestehenden Geschlechterrollen bei.

Eine Untersuchung oder Statistik, die sich mit geschlechterspezifischen
Angeboten in Bibliotheken befasst, gibt es bisher nicht, obwohl das
Konzept des Gender Mainstreaming eine Bestandsaufnahme der aktuellen
Situation vorsieht.\footnote{Pinl, Claudia: Gender Mainstreaming - ein
  unterschätztes Konzept / Claudia Pinl // In: Aus Politik und
  Zeitgeschichte, B. 33 -34 (2002), S. 3.} Dass eine solche Analyse
dringend nötig ist, hat Susanne Korb bereits 2008 festgestellt:
\enquote{Um den Gender Mainstreaming-Prozess zu initiieren,
fortzuentwickeln und Erfolge zu erzielen, ist es erforderlich, in
analytischer Weise den Ist-Zustand zu recherchieren, zu kennen und
öffentlich bewusst zu machen, um Handlungsoptionen daraus
abzuleiten}.\footnote{Korb, Susanne: Gender Budget -- Konzept und
  Bedeutung für das Management Öffentlicher Bibliotheken / vorgelegt von
  Susanne Korb. - Berlin, 2008. - 415 S. Online-Ausg.: Gender budget:
  Konzept und Bedeutung für das Management Öffentlicher Bibliotheken
  Berlin, Humboldt-Univ., Diss., 2008,~ S. 110.}

Sucht man nach Publikationen zu \emph{Gender} und Bibliotheken, stellt
man fest: Zu dem Thema wurde bisher wenig veröffentlicht. Auch auf~
Mailinglisten wie ForumÖB und InetBib wird es nur am Rande diskutiert,
genauso wie auf bibliothekarischen Kongressen. Der Mangel an
Thematisierung in der Fachwelt trägt dazu bei, dass für
Geschlechterrollen und den Mechanismen ihrer Reproduktion in
Bibliotheken kaum Sensibilität vorhanden ist.

\section*{Warum ist die mangelnde Gender-Sensibilität
problematisch?}\label{warum-ist-die-mangelnde-gender-sensibilituxe4t-problematisch}

\enquote{Als Frau wird man nicht geboren, zur Frau wird man
gemacht.}\footnote{Beauvoir, Simone de: Le deuxième sexe / Simone de
  Beauvoir. 47. éd.. --- {[}Paris{]} : Gallimard, 1950. - S. 13.}:
Simone de Beauvoir legte 1949 in ihrem Werk \enquote{Das andere
Geschlecht} erstmals dar, wie Geschlechterrollen nicht von Natur aus
angeboren, sondern soziale Prozesse sind -- und damit veränderbar. Das
biologische Geschlecht (\emph{sex}), das am jeweiligen Beitrag zur
potentiellen Fortpflanzung festgemacht wird, wird vom sozialen
Geschlecht (\emph{gender}) unterschieden. Es gibt also weder Interessen
noch bestimmte Leseverhalten, die Personen aufgrund ihres biologischen
Geschlechtes zugeschrieben werden können; es gibt lediglich Interessen,
die sie in ihrer Sozialisation als Junge bzw. Mädchen entwickelt
haben.\footnote{Rendtdorff, Barbara (Hrsg.): Geschlechterforschung :
  Theorien, Thesen, Themen zur Einführung / Barbara Rendtdorff\ldots{} -
  Stuttgart : Kohlhammer, 2011. - S. 220 ff.} Diese~ Interessen und
Verhaltensweisen, die von Menschen aufgrund ihrer
Geschlechtszugehörigkeit erwartet werden, werden als Geschlechterrollen
bezeichnet.

Für Bibliotheken ist an diesem Punkt entscheidend: Werden diese
Geschlechterrollen über\-nom\-men und reproduziert -- oder wird dies bewusst
vermieden?

\section*{Warum wir gendersensible Bibliotheken
brauchen}\label{warum-wir-gendersensible-bibliotheken-brauchen}

Auch wenn es heute -- anders als noch im Richtungsstreit um 1900 --~
einen Konsens darüber gibt, dass Bibliotheken ihre Leser*innen nicht
erziehen wollen, sondern sich vielmehr als Dienstleistungseinrichtungen
verstehen: Bibliotheken und deren Mitarbeiter*innen haben einen nicht
unwesentlichen Einfluss auf das Weltbild ihrer Benutzer*innen.

Zu ihren Aufgaben gehört es, in der Informationsflut Orientierung zu
bieten, indem sie eine Auswahl von Medien treffen und diese in einen
Ordnungszusammenhang (das heißt eine Systematik) stellen. Diese
Systematiken spiegeln seit jeher das Weltbild, in dessen Zusammenhang
sie entstanden sind, wieder: Man denke nur an Borges` chinesische
Enzyklopädie oder an die bibliothekarisch-bibliographische
Klassifikation in der DDR, die sich nach der Ideologie des
Marxismus/Leninismus ausrichtete. Untergliedern wir unsere
bibliothekarischen Systematiken mit Interessenskreisen, die wir nach
Geschlechtern benennen (\enquote{Frauen}, \enquote{Männer},
\enquote{Freche Frauen} etc.) oder stellen Regale zum Thema
\enquote{Liebe, Frauen, Familie} auf, suggerieren wir nicht nur, dass
das Interesse an bestimmten Medien geschlechtsspezifisch ist, sondern
reproduzieren auch die Rollen, die den Geschlechtern zugeordnet werden.
Für die Mädchen gibt es Bücher über Prinzessinnen, aus denen sie lernen:
es kommt darauf an, schön und empfindsam zu sein. Für die Jungen die
Bücher über Piraten: es kommt darauf an, stark und mutig zu sein.

Die Bibliothek unterstützt somit die Aufteilung von Interessen nach
Geschlecht, wenn sie Angebote speziell für Männer oder Frauen schafft.
Schon die Zusammenfassung der Literatur für \enquote{Frauen} in einen
Komplex mit \enquote{Liebe} und \enquote{Familie} suggeriert, dass
Familie und Liebe Themenbereiche sind, die ausschließlich für Frauen
relevant sind. Frauen wird unterstellt, sie hätten aufgrund ihres
Geschlechts ein verstärktes Interesse an Liebes- und Familienromanen;
gleichzeitig wird unterstellt, dass Männer genau diese Bücher nicht
lesen wollen. Ein Benutzer entdeckt vielleicht nie seine Leidenschaft
für Utta Danella, weil die Bibliothek deren Bücher mit dem
Interessenskreis \enquote{Frauen} versehen hat. Er fühlt sich
möglicherweise genauso wenig angesprochen wie das Mädchen, das gerne
Fußball spielt, aber von Anfang an auf die Pferdebücher aufmerksam
gemacht wird. Das spielerische Entdecken zuvor wenig beachteter
Themengebiete wird durch die hier beschriebene systematische Kopplung
von Geschlechterklischees und der Bestandsaufstellung erschwert:
wertvolle Effekte der Serendipität sind aus dieser Perspektive nahezu
unmöglich.

Auch wenn Benutzer*innen ein durch ihre gesellschaftliche Prägung
bedingtes geschlechtsspezifisches Literaturinteresse haben: Zweifelsohne
finden jene die von ihnen gewünschte Literatur auch dann, wenn diese
nicht in der Sachgruppe \enquote{Frauen} oder \enquote{Männer}
zusammengefasst, sondern thematisch aufgestellt ist -- im Jenaer
Beispiel also als \enquote{Liebesromane} oder \enquote{Familiensaga}.

\section*{Das geschlechtsneutrale Bibliotheksangebot: drei
Vorschläge}\label{das-geschlechtsneutrale-bibliotheksangebot-drei-vorschluxe4ge}

Die Bibliothek darf nicht Teil der Reproduktionskette von veralteten
Geschlechterrollen sein. Sie soll den Leser*innen das bieten, was sie
lesen wollen -- und zwar unabhängig von ihrem Geschlecht.

Welche Änderungen können wir also vornehmen, um das Bibliothekswesen in
Deutschland gendersensibler und geschlechtergerechter zu gestalten? Drei
Ideen:

\emph{1. Einen breiten Diskurs über das Thema \enquote{Gender in
Bibliotheken} in der Fachwelt schaffen.}

Wir brauchen eine Diskussion über den aktuellen Stand, die Relevanz des
Themas und Zielvorstellungen. Ein genderthematisches Panel auf dem
Bibliothekskongress wäre ein guter Anfang, gefolgt von Diskussionen in
Fachzeitschriften, Mailinglisten, Blogs und internationalem Austausch.

\newpage 

\emph{2. Das Thema Gender in die bibliothekarische Ausbildung
integrieren, Fortbildungen anbieten.}

Gender ist ein Querschnittsthema -- als solches muss es auch in den
bibliotheks- und informationswissenschaftlichen Studiengängen (zum
Beispiel beim Bestandsaufbau oder bei der Literaturvermittlung)
integriert werden. Auch ein Modul \enquote{Gender und Diversity in
Bibliotheken} ist denkbar. Um Bibliothekar*innen, die im Beruf stehen,
die Auseinandersetzung mit \emph{Gender} in Bibliotheken zu ermöglichen,
muss es zudem entsprechende Fortbildungsangebote geben.

\emph{3. Take action: Interessens-, nicht geschlechtsspezifische
Angebote schaffen.}

Liebesromane, Familiensagas, Prinzessinnen- und Piratenbücher, Literatur
für alle!

\section*{Literatur}\label{literatur}

Beauvoir, Simone: Le deuxième sexe / Simone de Beauvoir.~--- 47. éd..
--- {[}Paris{]} : Gallimard, 1950.

Korb, Susanne: Gender Budget - Konzept und Bedeutung für das Management
Öffentlicher Bibliotheken / vorgelegt von Susanne Korb. - Berlin, 2008.
- 415 S. Online-Ausg.: Gender budget: Konzept und Bedeutung für das
Management Öffentlicher Bibliotheken. Berlin, Humboldt-Univ., Diss.,
2008-

Pinl, Claudia: Gender Mainstreaming - ein unterschätztes Konzept /
Claudia Pinl // In: Aus Politik und Zeitgeschichte, B 33 -- 34 (2002),
S. 3-5.

Rendtdorff, Barbara (Hrsg.): Geschlechterforschung : Theorien, Thesen,
Themen zur Einführung / Barbara Rendtdorff\ldots{} - Stuttgart :
Kohlhammer, 2011. - 237 S.

%autor

\end{document}
