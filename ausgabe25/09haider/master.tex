\documentclass[a4paper,
fontsize=11pt,
%headings=small,
oneside,
numbers=noperiodatend,
parskip=half-,
bibliography=totoc,
final
]{scrartcl}

\usepackage{synttree}
\usepackage{graphicx}
\setkeys{Gin}{width=.6\textwidth} %default pics size

\graphicspath{{./plots/}}
\usepackage[ngerman]{babel}
%\usepackage{amsmath}
\usepackage[utf8x]{inputenc}
\usepackage [hyphens]{url}
\usepackage{booktabs} 
\usepackage[left=2.4cm,right=2.4cm,top=2.3cm,bottom=2cm,includeheadfoot]{geometry}
\usepackage{eurosym}
\usepackage{multirow}
\usepackage[ngerman]{varioref}
\setcapindent{1em}
\renewcommand{\labelitemi}{--}
\usepackage{paralist}
\usepackage{pdfpages}
\usepackage{lscape}
\usepackage{float}
\usepackage{acronym}
\usepackage{eurosym}
\usepackage[babel]{csquotes}
\usepackage{longtable,lscape}
\usepackage{mathpazo}
\usepackage[flushmargin,ragged]{footmisc} % left align footnote

\usepackage{listings}

\urlstyle{same}  % don't use monospace font for urls

\usepackage[fleqn]{amsmath}

%adjust fontsize for part

\usepackage{sectsty}
\partfont{\large}

%Das BibTeX-Zeichen mit \BibTeX setzen:
\def\symbol#1{\char #1\relax}
\def\bsl{{\tt\symbol{'134}}}
\def\BibTeX{{\rm B\kern-.05em{\sc i\kern-.025em b}\kern-.08em
    T\kern-.1667em\lower.7ex\hbox{E}\kern-.125emX}}

\usepackage{fancyhdr}
\fancyhf{}
\pagestyle{fancyplain}
\fancyhead[R]{\thepage}

%meta
%meta

\fancyhead[L]{M. Metzendorf, A. Kellersohn \\ %author
LIBREAS. Library Ideas, 25 (2014). % journal, issue, volume.
\href{http://nbn-resolving.de/urn:nbn:de:kobv:11-100219308
}{urn:nbn:de:kobv:11-100219308}} % urn
\fancyhead[R]{\thepage} %page number
\fancyfoot[L] {\textit{Creative Commons BY 3.0}} %licence
\fancyfoot[R] {\textit{ISSN: 1860-7950}}

\title{\LARGE{How library and information science can save the world and why to care}} %title %title
\author{Jutta Haider} %author

\setcounter{page}{82}

\usepackage[colorlinks, linkcolor=black,citecolor=black, urlcolor=blue,
breaklinks= true]{hyperref}

\date{}
\begin{document}

\maketitle
\thispagestyle{fancyplain} 

%abstracts

%body
\emph{held as a keynote lecture at the Bobcatsss conference 2014 in
Barcelona}

I think we agree that this is quite a title, quite a promise I have to
live up to. Well, the truth is I tricked you. Actually, there was a
subtitle, but then I thought I would really like to have at least a few
people in the audience, so I decided that if ever there was a time for
being bold then this was it and I dropped it. Then of course -- as you
would -- I had second thoughts and well: Here it is -- the full title is
actually: \enquote{How library and information science can save the
world and why to care. \ldots{} Or at least why we should let people
know what we know about that unruly thing called information and how
this might possibly contribute a little bit to help preventing things
from getting even worse.}

This complicates things. You probably would not even have bothered
getting up today for that much vagueness. It is -- sort of -- almost the
same title, but it is full of maybes and academic blurriness -- which is
good and which is bad.

\section*{Making things
complicated}\label{making-things-complicated}

We -- I mean we library and information science people, scholars and
practitioners alike - are excellent at making things complicated, at
problematizing, at defining and redefining our subject - information -
and thinking about our discipline. And obviously, it is complicated, and
being able to see and to articulate complexities, ambivalences, and
connections and doing so with respect for others and other disciplines,
professions and knowledge traditions is a very valuable ability.

Meta-epistemic ability, as it could be called for this purpose and as
most of you have, can be used to describe the ability to reflect on
differences between disciplines and on which expressions these
differences can take. In fact, we are specifically trained in this. It
is a required skill for working with information in libraries and in
other similar institutions. Meta-epistemic ability is not least
fundamental for working within different subjects and areas and most
importantly for working with different users with divergent needs and
requirements. After all, organising knowledge and retrieving information
and communicating it to people, to users, are all subjects, that can be
described as core aspects of the field. And they can only be done with a
focus on exactly that -- difference. To illustrate this: In order to
classify in a systematic way you have to be able to see similarities and
differences between individual documents and groups of them. So being
skilled at seeing differences and making something with it is a
pre-requisite for all you and all of us, for being successful in our
field, professionally and in research. As the famous definition by the
anthropologist, Gregory Bateson (1988), has it --- loosely paraphrased:
Information itself is a difference, but more interestingly -I find -
Information makes a difference.

We are so good at making things complicated, at looking for differences
and missing bits, that we sometimes forget how obviously relevant
library and information science is for so many issues that concern
society and the world and how -- just by virtue of its name -- obviously
relevant it must seem to outsiders and how we should obviously claim a
central position with at least some measure of arrogance.

\section*{A field at crossroads}\label{a-field-at-crossroads}

Before moving on, I need to apologise for navel gazing, for doing what I
just criticised. After all, it is important to remind us and others that
information, and with it its study and the study of its institutions, is
always for something. It is societally relevant in a way that is easy to
communicate, that we have experience in communicating, that many accept
as societally relevant and not least that is needed. Not infrequently
though this is motivated by the need to develop better information
systems. Which is in itself neither wrong nor unnecessary, but it is
important to keep in mind that information systems are always just means
to an end, never an end in themselves.

Still, as much as we may want otherwise, library and information science
is more akin to field of research combined with a host of professional
educations than to a discipline in a more traditional sense. This is not
surprising considering that we have in most countries not been around as
a research discipline for that long. That is not unique in any way, and
we are in fact not so different from say media and communication
studies, gender studies, science and technology studies, the educational
sciences or even economics. All these are rallying around hard to define
concepts, integrating different methods, perspectives, theories -
sometimes more and sometimes less fruitful, sometimes more and sometimes
less visible to the outside. It might be an advantage, it might be a
disadvantage. It is any way not our privilege, we are far from being the
only ones. In times of budget cuts - being interdisciplinary might put
you in a disadvantaged position - as some have pointed out
(e.g.~Buckland, 2012) - since there is a conservativeness ruling the
allocation of funds in academia that does not usually benefit the
hard-to-defines, the interdisciplinary ones, the up-and-comings.

On the other hand, disciplines are never fixed, they are born, they
grow, they change, they are different things to different people, they
are the results of university politics and of research policies,
publishing traditions -- and even library collections play a role here.
While in itself extremely interesting to investigate, from our point of
view, we need to be pragmatic. Library and information science exists,
it exists as one, -- as conferences such as this are testimony to and as
educations such as yours are a part of -- and that is a very good thing.
It exists at cross-roads. Also this is a very good thing and quite
special. The collected experience of being required to relate to other
fields, to incorporate other research traditions, to adapt methods,
theories and perspectives, to a degree even to question yourself, is
huge and very valuable. The experience of being situated between
research, professions, and policy makers, especially, cultural policy
and information policy institutions is invaluable as is the
professionals experience of liaising between the public, citizens and at
times other users and political and societal requirements.

There are downsides as well, I admit, but now I want to focus on the
positives and filter the negatives through pink sunglasses. I am not
doing this just for the purpose of creating a feel-good atmosphere, but
rather to set the scene for what I have to say about environmental
information, information for sustainable living, later on and about how
much library and information science or information studies are needed
in this area of research.

And here it becomes interesting to ask ourselves: \enquote{What is
information science, what is library and information science?} Or
actually even more so: \enquote{What is it we are investigating? - And
how?} Or as Michael Buckland (2012) formulated it most recently
\enquote{What kind of science can information science be?} There are of
course many answers to that and many good ones. I will try my best at
giving one -- with the help of Michael Buckland (1991; 2012) --, before
moving on to talk about information in practice (cf.~Cox, 2013), more
specifically about everyday life information on environmental protection
and how to live more sustainably - my own area of research in recent
years.

\section*{Information is an unruly thing? Why unruly? And why a
thing?}\label{information-is-an-unruly-thing-why-unruly-and-why-a-thing}

If we return to the secret title that I initially withheld from you, I
deliberately chose to describe information as an \enquote{unruly thing}.
Why is that? Well, firstly, as we all know, there is no agreed view on
what information actually is and many scholars in LIS and other fields
have tried to define it, carve out its different borders, find out what
it not is and so forth (for a recent overview see e.g.: Hjorland, 2013).
Some had the ambition to find the one answer, to develop a single theory
and definition of information. Others -- often coming from a more
constructionist angle -- are more careful and have tried to come to
terms with how we can find a concept we can all rally around and use,
but which is still flexible and adaptable enough to accommodate
different interests and needs to suit a field like ours, multi-facetted
both in research and in its professions. I find this second approach
more fruitful and situate my own research there. In fact, having a
shape-shifting concept that furthermore can fruitfull be investigated
from a very wide array of different angles, perspectives and methods, at
the centre of our field, makes the field not only exciting, it is also
what makes it - I suggest - worthwhile and relevant to continue
researching and working with it.

One such an approach, and a particularly interesting and successful one
-- also a very pragmatic one -- is Michael Buckland's (1991; 2012)
approach, where he tear apart three ways of understanding information:
(1) information as knowledge, (2) information as a process (3)
information as a thing.

\section*{What does that mean?}\label{what-does-that-mean}

\emph{Information as knowledge} is probably the most common conception
and perhaps the one that makes the most problems. What Buckland means
here, and I can subscribe to that notion, is that information can be
conceptualised as knowing, \emph{as a result of being informed} rather
than merely knowing of or knowing that. He suggests a constructed notion
of information as knowledge, far removed from the ideals of analytical
philosophy framing information/knowledge as justified true belief, which
puts all focus on and claims to truth of a piece of knowledge. Buckland
in fact takes issue with both parts of that understanding, with
justified and with truth and lands in a what could be called -- and this
is my addition -- a pragmatic material semiotic notion of information as
knowing as constructed, as relational, as culturally specific, and with
the distinction between belief and truth entirely blurred or
meaningless. Most importantly, this is a notion that means we are
dealing with questions of trust rather than truth; an imperfect notion
in every sense, so to speak, but one that works.

Let's move on to the notion of \emph{information as a process}. This is
about getting informed, about learning, about making sense. Here, in
order to sit well with this just described notion of knowledge as
imperfect, as cultural and social, it is of importance to ground our
understanding of learning in theories of education, of communication,
that account for the social and incremental character of learning.

Finally, \emph{information as a thing}, the third of Buckland's notions
of information. Here he draws mainly on document theory to conceptualise
the materiality of information. That is the way in which information is
always also \emph{in} and \emph{through} something and this something,
be it a book, a poster, a website, a building or Suzanne Briet's (2006
{[}1951{]}) famous antilope, co-determines or is part of information.
The material condition affords a certain information and if we
acknowledge the social structuring of our material world, then again the
social, the cultural, is also in this notion of information re-inscribed
back into information itself. Let's make this the focus of the next part
of my presentation.

But before that, I want to give you an example that ties back to
sustainability: Information as knowledge. We have knowledge of
scientific research that ties the use of fossil fuels, oil and gas, to
increased CO2 levels in the atmosphere and hence to climate change. We
have no way of knowing if this true or not, the only thing we can go by
is trust in science and trust in institutions that validate science and
make policy. We might have some anecdotal evidence on storms, on changes
in how much it rains, whatever, but we have to trust institutions that
this should be linked to human induced climate change. Information as a
process then? How did we get to know? We have been informed in some way
or another, through formal education, through the news, through
information campaigns by environmental organisations, you name it.
Mostly this happened in informal ways, surrounding us. Materiality then?
This shifts of course, but if we take a device-centred approach, one
that accounts for how things, devices are what they are in a context,
and carry meaning -- they are documents, so to speak -- then for
instance by now airplanes, cars or even carbon calculators have become
part of climate change in a way that makes them documents that tell us
something about it. We will see other examples later on, where the issue
is a bit more humble -- \emph{just} environmental protection or
sustainable living.

\section*{Materiality of informing}\label{materiality-of-informing}

We spend quite some time trying to tease apart different notions of
information. And now I will throw them all together again. I am not
doing this to make a mess and certainly not to demonstrate that we do
not need or that we do not need to reflect on these different
understandings, but rather to complicate things by drawing a thread
through these notions in a way that is productive for what I have to say
about everyday life environmental information.

Information is not something that simply \emph{is} -- it is something
that \emph{happens} in things, in processes, in cultures. And if we
complicate Buckland's beautiful division of labour even further, then we
can add: it is something that happens in material forms of some sort or
another, in doings, in practices, in documents. To put it differently:
Information happens in practices when people meet documents (cf.~Haider
2012). Information is furthermore about creating meaning and through
this it is about relations, between documents, things, people, and
doings.

This focus is unique for our discipline -- focus on information in its
material form -- in encyclopaedias, in databases, in libraries, in
museums, in antilopes and even in compost heaps as I have argued in my
own research, and in the social practices connected to these things.~
This is in contrast to how information is seen in many other areas, such
as for instance cognitive sciences or even epistemology where
information is dematerialized and of itself and on its own. This is a
notion which suits the purposes of epistemology and cognitive sciences
very well, but which might only be of limited use for library and
information science or information studies and for what we actually do,
in professional practices and also in what we research.

\section*{Environmental
information}\label{environmental-information}

In my recent research on environmental information and sustainable
living I have tried to draw together theories of practices and theories
for understanding individualized forms of civic engagement. Both are
situated within what Library and Information Science scholar Bernd
Frohmann (2004) describes as a discourse of practices, where knowledge
is seen as manufactured and assembled of diverse practices; material,
social, cultural, or discursive. He puts this in contrast to
epistemological discourse, where knowledge is discovered, immaterial and
talked about in terms of truth, meaning or representation. Also this
notion cuts across Buckland's different understandings of information,
yet without contradicting it.

Throughout this talk, I have mentioned the term \emph{practice} a few
times, without further ado, but now since I already introduced
Frohmann's discourse of practices, I want to throw a definition at you.
In the words of Andreas Reckwitz (2002, p.250) we can phrase it like
that: \enquote{A practice is {[}\ldots{}{]} a routinized way in which
bodies are moved, objects are handled, subjects are treated, things are
described and the world is understood.} I leave it at that -- for now,
but I will return to this notion of practice as something different than
behaviours, actions, doings, namely as a routinized and culturally
specific way of connecting activities, things, people and meaning -- in
a little while.

When researching environmental information I have put much effort into
highlighting the materiality of information, the materiality of
informing together with the social character of the material. This is
very tangible when looking at how environmental information is made
sense of. How does it arrive where it is supposed to arrive? There seems
to exist a quiet agreement that information is important for how we
behave, what we do, what we consume, how we live, but also for how
policies are made, how institutions and how companies make decision. And
this agreement also applies to the environment and to environmentally
friendly living.

Before I continue I need to make a disclaimer: It is obvious, politics
matter here. Clearly, I am not suggesting that we need to shift all
responsibility on the individual and make environmental protection a
question of private engagement alone to solve the large challenges we
are facing in the area. Political commitments, policies, and
institutional change are required and arguably larger socio-economical
shifts. However, this does not mean that what people actually do and how
they reason about their doings is less relevant, less interesting and
hence certainly not less worthy of research.

Obviously, you would say, if you know what a certain behaviour leads to
then you will act accordingly. But this is seldom the case. Just think
about any health issue of your choice -- eating junk food, excessive
drinking, smoking, contraception, exercising -- you name it. The
relation between what we know and what we do is very far from
straightforward. We have received the information that was targeted at
us, we made sense of it, we learned and then we more often than not
totally dismiss it -- or do we? I leave that open for now -- because it
is more complicated than that and I am not a cynic. Yet still, these are
all issues that relate to our own body, very concretely and after a
while not very abstract at all. Information relating to how to lead more
environmentally friendly lives is in an even worse position here -- it
is abstract, the negative consequences are not felt by us -- yet,
directly, on our bodies. They relate to changes that are then expressed
in numbers, in degrees Celsius, in extinct animals, in levels of carbon
dioxide, or they are emotionally and geographically distanced from us --
that is they happen to \emph{other} people in remote places, not felt in
weight gain, in loss of shape, in a headache for instance.

I want to illustrate this with some examples from interviews taken from
my own research on how people make sense of information on how to live
more environmentally friendly, more sustainably in their everyday lives.
We can see that this task of sense-making is not an easy one, when it
has to be connected to some abstract threat, a distant future or an
abundance of guidelines.

\begin{quote}
Person 1: {[}\ldots{}{]} this is technically quite difficult; you should
tell us about what the EU should do with its climate deal and that you
should reduce 20\% of this and that. This is quite removed from concrete
demands.
\end{quote}

\begin{quote}
Person 2: What happens often is that when something particular happened
\ldots{} I mean hurricanes or natural disasters \ldots{} then it comes
to the fore, but in my day-to-day life it's not like I go around and am
concerned. And it's also not so that I, for instance, refrain from
flying just because of environmental reasons or so, but it is, well,
what shall I say\ldots{}
\end{quote}

\begin{quote}
Person 3: I was in Stockholm to visit my brother and we sat at NK {[}an
inner-city shopping mall{]} and I saw a small leaflet and there they
showed the different labels' /\emph{inaudible}/ and there was an A4 page
with 20 to 30 labels and I just shook my head. That is too much, too
much information and in a way they are fighting against each other and I
think that's a shame {[}\ldots{}{]}. In a way, each organisation wants
to promote itself and in a way this is done at the expense of others.~
\end{quote}

\begin{quote}
Person 4: We have an understanding for environmental problems,
sympathise with the idea, but it's not really present. {[}\ldots{}{]}
And a little bit of {[}recycling{]}, we do that even out here. We have a
compost heap and I find that natural in a way, nature's way of taking
care of waste. But really, I don't have any knowledge on environmental
questions.
\end{quote}

You can see, these are four different people expressing more or less the
same feeling of abstraction and alienation, of irrelevance to their
lives in different ways. All four, also in different ways, show they do
in fact know a lot about the environment, about what they have been told
is the so-called correct way of doing things, they exemplify: flying,
consumption, recycling, connect to politics and policy making (the EU),
to consequences, natural disasters that is, to different interests and
that includes commercial ones. They reflect on their difficulties, their
indifference, their coping strategies you name it. They show knowledge,
awareness, and they reflect on the difficulty of, well, practising
information. The divide between abstract information and practice is not
due to a lack of knowledge. It is not even due to a lack of awareness of
a possible link between individual practices and potential global
consequences. The missing link seems to lie with its actualisation in
everyday life practices. And this in turn implies that the question what
work information campaigns can \emph{do}, has to be asked afresh.

And now let's look at some more quotes, from the same interview study,
the quotes are partly from the same people. Here things get a lot more
concrete, a lot more anchored in everyday life and suddenly the
perspective changes and material things and what you do with them
concretely comes into focus.

\begin{quote}
Person 5: We are quite economical with water, really. We don't do
anything huge. {[}\ldots{}{]} You do it's like that because that's a way
of conducting oneself that you have here and\ldots{} We have a small hot
water boiler, so you can't really shower for that long.
\end{quote}

\begin{quote}
Person 2: It's a bit like that, it's summer and you want to have a good
life. And then you drive to {[}\ldots{}{]} and buy fresh rolls. But it
can be like that there are 50 cars driving from the summer village to
{[}\ldots{}{]} to -- I don't know -- to buy bread rolls.
\end{quote}

\begin{quote}
Person 4: It's just easier because it is a residential area
{[}\ldots{}{]} quite defined in a housing body. There are many flats and
no distances {[}\ldots{}{]}. So it's easier to organise this. I don't
have to go very far for putting my waste into the right bin.
\end{quote}

\begin{quote}
Person 6: It's nothing you reflect on that you should not\ldots{} that
now we will ride the bike to the beach and leave the car. It's not like
that. I don't think like that, but I think now we will take us to the
beach or to a wood somewhere. For this, we take the bikes.
\end{quote}

What is interesting here, leaving aside the variety of activities people
mention as relevant for the environment, what I want you to focus on, is
the way in which these are connected -- sometimes quite explicitly and
reflectively -- to routines, to contextual circumstances, to the
expected, the norm -- you name it, but to something beyond a conscience
doing, even when the practice is then connected to the environment in an
after-construction as this interview situation.

Now I used the term \emph{practice} again. What is that? Let's quickly
revisit Andreas Reckwitz's (2002) definition that we introduced earlier
and which says \enquote{A practice is {[}\ldots{}{]} a routinised way in
which bodies are moved, objects are handled, subjects are treated,
things are described and the world is understood.} These are ways in
which we do things with things, ways which are shared between people and
which are culturally specific. They have to be quite clearly
circumscribed and be repeatable and routinised. Basically, if something
is done just once it cannot be a practice in this theoretical
understanding. A practice is social, it is code, it is \emph{the
culturally agreed} way in which something is done. It is the
\emph{normal} way. If something is done differently it sticks out. For
instance, since this conference takes place in Spain and has many
international visitors, a good example can be taken from travelling: It
is the normal way to fly from Sweden to Spain, anything else you have to
explain. Another example: As a researcher it is normal practice to
participate in academic conferences. This involves giving a talk,
presenting a poster or participating in a panel or meeting. Just
attending a conference as a researcher without presenting or
participating in something or other is considered outside the norm and
people will ask about it. Also, you will find it hard to get university
funding. So in this case, a paper that you have submitted is not only a
paper where you communicate the content of the research, it is also part
of a set of practices and the entrance key to the regular door of a
conference.

\section*{Doing change}\label{doing-change}

Practices can be recognised as practices because they are stable -- at
least sort of -- and here we have the problem. We can talk about them
because they are constant, they are the norm, they are what we do not
think about when we do it. Yet information, the unruly thing that
happens when people meet documents, is about difference, about change.
This puts us in a difficult place; but only at first sight.

Obviously practices do change, they have always done that. As not least
Elisabeth Shove, who has researched environmentally relevant practices
at length, has shown normality and especially normal ways of doing
everyday life things with very common objects for quite mundane
purposes, are not at all stable, they are in constant flux (e.g.~Shove
and Spurling (eds.) 2013; Shove 2004). Just think about, for instance,
showering, cooking, doing the laundry --~all of these are quite
significant for the environment. These are seemingly simple practices
which have very stable cultural meanings and positions in our lives in a
way, but if we look closely how we do these things today, we can see it
is quite different from how our parents did them just 30 years ago and
really not to be compared with how it was done, say, 80 years ago.
Showering daily was just absurd then, and being able to use deep-frozen
vegetables and South-American meat on a daily basis for your casserole
was hard to imagine, not to talk of washing your clothes after having
worn them just once, which we do quite often nowadays. We are still
doing the laundry, but how often we do it, how effectively we do it,~
when we think it is necessary to do it, and with this how much energy it
takes and how much detergent we use, has changed dramatically. We are
still cooking family meals, but how we we do it, what we use for doing
it, how long it takes and not least how much energy it consumes, cannot
be compared to how it used to be. What all these examples show, as
Elizabeth Shove and her colleagues have investigated so beautifully, is
that \enquote{materials are integral to doing} (Shove et al., 2007,
p.67). The microwave, the washing machine, the tumble dryer, the
freezer, the power shower, cheap electricity and so forth, all these
shape the practices we live by and what we consider normal in our
culture and together they change, and \enquote{materials and practices
co-evolve} (ibid. p.66).

And this is, I think, what we see in how my interview partners reflect
on their lives and its environmental impact:

\begin{quote}
Person 5: We are quite economical with water, really. We don't do
anything huge. {[}\ldots{}{]} You do it like that because that's a way
of conducting oneself that you have here and\ldots{} \textbf{We have a
small hot water boiler, so you can't really shower for that long}.~
\end{quote}

\begin{quote}
Person 4: \textbf{It's just easier because it is a residential area}
{[}\ldots{}{]} quite defined in a housing body. There are many flats and
no distances {[}\ldots{}{]}. So it's easier to organise this. \textbf{I
don't have to go very far for putting my waste into the right bin.}
\end{quote}

\begin{quote}
Person 6: It's nothing you reflect on that you should not\ldots{} that
now we will ride the bike to the beach and leave the car. It's not like
that. I don't think like that, but I think now we will take us to the
\textbf{beach} or to a \textbf{wood} somewhere. For this, we take the
\textbf{bikes}.
\end{quote}

It is circumstantial, it is context, it is also materials that define
whether you take longer or shorter showers, whether you recycle, whether
you drive a car or bike. It is not -- in these cases at least -- a
conscious decision. When it is too conscious it is not a practice, even
-- at least not yet -- and then it's not very likely to be sustained.
Sustainability requires routine and routine is what is normal. This
illustrates also quite clearly that what is at issue is not simply
connecting a private practice to a global issue, or the other way
around. This is something, which seems easily done, even if often
connected with an overwhelming sense of powerlessness or guilt. The
issue might be constructing this initially merely private practice as
constructive and as directly meaningful for a person in a certain
context; to transform a feeling of guilt and alienation into positive
and enacted commitment. And this is, I suggest, where information and
its transformative capacity might come in, but we have to understand
that information comes through social practices and lands in devices, in
things that are meaningful in specific cultures.

How can we, scholars, professionals, students, then inform about these
issues in a way that also leads to some change in a direction that
minimises damages. I do not have the answer, just some of the questions,
but collectively as a field of research and as a field of practices we
have something important to contribute. Other disciplines talk about and
research information on the environment and related -- my interest -- to
practices of everyday life (e.g.~Bartiaux, 2008; Ek \& Söderholm, 2010;
Vittersø, 2002). But information studies or library and information
science is largely missing. This is not just a shame for us, because we
miss out funding opportunities and options to make our competence
visible, this is first and foremost a shame for the interdisciplinary
and highly relevant field of environmental studies. We know a lot about
information as knowledge, as a process and as a thing and so many times
researchers from the field and practitioners have put the finger on the
deeply embedded character of information, on the need to ground it in
practices, behaviours, whatever you call it, in what people actually do
and in ways in which these doings are culturally shaped -- on it being
always in the making and never being just a simple equation, on the
unruliness of information.

There are clearly different kinds of what can be described as
environmental information. The kind I have investigated and which I also
refer to here is that which so-called \enquote{ordinary} people and
policy makers connect to everyday life, to leading more environmentally
friendly lives. I deliberately say that which people connect to leading
environmentally friendly lives, rather than which is explicitly
\emph{about} how to lead such lives. Since, while targeted campaigns
might have this '\emph{aboutness'}, it is not always those that arrive
in people's practices or even in their awareness. In fact, the
disconnect between what people know about the environment, or what they
say they know and also reflect on, the connection between certain
practices and objects and environmental destruction and protection and
what they actually do -- is striking (e.g.~Haider, 2011).

And I am not saying this to a make a moralistic point -- my own
knowledge is strikingly far removed from my own doings -- I am saying
this to make a claim for the need of information studies or library and
information science perspectives in the area of environmental studies.
The connection between knowing and doing is highly problematic. And
information is a link between them -- yet if we investigate information
purely as epistemic content, as facts and figures, as how to-s, as that
which fills a gap and impacts behaviours, then the only thing we are
likely to find is that it does not do anything that it does not make a
difference after all and that people simply do not care about much
beyond their own direct needs. But this simply means not looking hard
enough. This does not account for the way, in which it has also been
shown that people learn, that they make sense, that they can explain and
make connections, yet it does not trickle down into behaviours --
directly. It does, but slowly and in different places than we might
expect.

It is a connection that, I suggest, fruitfully can be visualised as
\emph{information} in the sense as developed earlier; as that meeting
between people and documents in social practice which potentially makes
a difference. Yet, this information is notoriously hard to study and
notoriously hard to capture. It is an unruly thing, but one which we
know a lot about.

For instance we know, from research in other empirical fields -- from
contraception (e.g.~Rivano Eckerdal, 2012) to learning (e.g.~Sundin \&
Francke, 2009), work practices (e.g.~Lloyd, 2009), photography
(e.g.~Cox, 2013) and health (e.g.~McKenzie, 2010) -- that information
practices build on intricate relationships between objects and ways of
speaking. We know that information activities can be anything and
anywhere. They can be very targeted, but often they are not. They happen
rarely in isolation, they are fun and they are necessities, they just
happen to people and they are initiated by them. They are everywhere,
they are about facts and they are about feelings, they are part of work,
of schools, of being a citizen, a consumer, a family member. They are
one big mess. Most importantly, they are intrinsic parts of other
activities, they are embedded in larger social practices and shaped by
these. In turn of course, information also shapes practices and this is
what is hoped for when attempting to change behaviours with information.
And this is what requires a deep understanding of how information might
do this.

Scholars from our discipline and from related fields have shown the many
subtle ways in which this is the case. What people know, what they have
learned through their various conscious and unconscious information
activities is rarely directly connected to what they do. There are
connections, but finding those, studying those, shaping those, and
simply taking those seriously and not doing away with it as hypocracy or
lack of self-control, is not an easy task. But it is what \emph{we} do
and nobody else does it as well as we.

\section*{Concluding remarks}\label{concluding-remarks}

I want to make a call for a research agenda and for a practice agenda
using a library and information science torch to look at environmental
issues; for instance how information on how to do environmentally
friendly living is shaped, consumed, made sense of, how information
shapes policy, how our own institutions embed environmental values, how
information campaigns are made, how young people develop values
connected to information on environmental issues and so forth.

Environmental problems and climate change are amongst the greatest
challenges our society faces. Information is not the key to solving
those, far from it, but it is important. Yet, we~ know so much about
information and its unruliness do not contribute in any significant way.
Why? How can it be that Library and Information Science, which so often
is motived by its social relevance, is almost entirely missing from
environmental research -- a field that in itself is a multidisciplinary
research area with contributions from many disciplines? I do not know
the answer to this rhetoric question. Yet it is important to remember
that while the improvement of information systems in all their shapes is
certainly a worthwhile endeavour, what these systems actually are
\emph{for} has to be our starting point. We need to contribute to the
development of social and cultural understanding of information and
information systems.

Now I have almost reached the end of my presentation, just one more
thing which I have promised in my rather bold title: How Library and
Information Science can save the world and why to care? Well, actually
here I fooled you again. I do not have the answer. However, I can say
that much: You are in a unique position. You have the skills, the
knowledge, you represent a relevant discipline that can make a much
needed contribution, that has a tradition of social relevance and a
tradition of connecting different areas of research, fields of
application, different institutions and the academy and policy making.
It would be a shame not to use this position. Library and Information
Science can actually make a difference.

\section*{References}\label{references}

Bartiaux, F. (2008), Does environmental information overcome practice
compartmentalisation and change consumers' behaviours? \emph{Journal of
Cleaner Production}, 16 (11), 1170-1180.

Bateson, G. (1988). \emph{Mind and nature: A necessary unity}. Toronto;
New York : Bantam Books.

Buckland, M. K. (1991). Information as thing. \emph{Journal of the
American Society for Information Science and Technology}, 42(5),
351-360.

Buckland, M. (2012). What kind of science can information science be?.
\emph{Journal of the American Society for Information Science and
Technology}, 63(1), 1-7.

Briet. S. (2006 {[}1951{]}). What is documentation? In: \emph{What is
documentation? English translation of the classic French text.
(Translated and edited by Ronald E. Day and Laurent Martinet with
hermina G.B. Anghelescu.}Lanham; Toronto; Oxford: Scarecrow Press,
9-46.~

Cox, A. (2013). Information in social practice: A practice approach to
understanding information activities in personal photography.
\emph{Journal of Information Science}, 39(1),61-72.

Ek, K., \& Söderholm, P. (2010). The devil is in the details: Household
electricity saving behavior and the role of information. \emph{Energy
Policy}, 38(3), 1578-1587.

Frohmann, B. (2004). \emph{Deflating information. From Science Studies
to Documentation}. Toronto: University of Toronto Press.

Haider, J. (2011). The environment on holidays or how a recycling bin
informs us on the environment. \emph{Journal of Documentation}, 67(5),
823-839.

Haider, J. (2012). Interrupting practices that want to matter: The
making, shaping and reproduction of environmental information online.
\emph{Journal of Documentation}, 68(5), 639-658.

Hjørland, B. (2013). Information science and its core concepts: Levels
of disagreement. In \emph{Theories of Information, Communication and
Knowledge}. Netherlands: Springer, 205-235.

Lloyd, A. (2009). Informing practice: information experiences of
ambulance officers in training and on-road practice. \emph{Journal of
Documentation}, 65(3), 396-419.

McKenzie, Pamela (2010). Informing relationships: small talk, informing,
and relationship building in midwife-woman interaction.
\emph{Information Research}, 15(1).

Reckwitz, A. (2002). Toward a theory of social practices. A development
in culturalist~theorizing, \emph{European Journal of Social
Theory},5(2), 243-263.

Rivano Eckerdal, J. R. (2012). Information sources at play: The
apparatus of knowledge production in contraceptive counselling.
\emph{Journal of Documentation}, 68(3), 278-298.

Shove, E. (2003). \emph{Comfort, cleanliness and convenience. The social
organization of normality}. Oxford: Berg.

Shove, E., \& Spurling, N. (Eds.) (2013). \emph{Sustainable practices:
Social theory and climate change.}London and New York: Routledge.

Shove, E., Watson, M., Hand, M., \& Ingram, J. (2007). \emph{The design
of everyday life (Cultures of consumption)}. Oxford: Berg.

Sundin, O., \& Francke, H. (2009). In search of credibility: pupils'
information practices in learning environments. \emph{Information
Research}, 14(4). paper 418. Available at:
\url{http://InformationR.net/ir/14-4/paper418.html}.

Vittersø, G. (2002). \emph{Environmental Information and Consumption
Practices. A Case Study of Households in Fredrikstad}. Oslo: Sifo.
Available at:
\url{http://www.sifo.no/files/file48548_fagrapport2003-4.pdf}.

%autor

\end{document}
